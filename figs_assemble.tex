%\documentclass[a4paper,10pt]{article}
% \documentclass[a4paper,10pt]{scrartcl}
% \usepackage[margin=0.5in]{geometry}
% 
% \usepackage[utf8]{inputenc}
% 
% \usepackage{wrapfig}
% \usepackage{lscape}
% \usepackage{rotating}
% 
% \usepackage{amssymb}
% \usepackage{hyperref}
% \usepackage[square]{natbib}
% 
% %fancy stuff
% %\usepackage{datetime}
% \usepackage[fleqn]{amsmath}
% \usepackage{helvet,mathptmx}
% \usepackage{caption}
% \usepackage{subcaption}
% \usepackage{xcolor}
% \usepackage[normalem]{ulem} % striked-out text
% 
% \usepackage{verbatim}
% %\usepackage{listings}
% 
% 
% \title{Boundary conditions of the SNS setup:\\switching from REMO to CLM \& new current velocities}
% \author{Onur Kerimoglu}
% \date{\today}
% 
% \pdfinfo{%
%   /Title    (Boundary conditions)
%   /Author   (Onur Kerimoglu)
%   /Creator  (Onur Kerimoglu)
%   /Producer ()
%   /Subject  ()
%   /Keywords ()
% }


% GENERAL MANUSCRIPT TEMPLATE

	% by Nicholas E. Reith

	% DATE: November 26th, 2014

%--------------------------------------------------------%
%	PREAMBLE
%--------------------------------------------------------%

% DOCUMENT CLASS
    % Change "letterpaper" to "a4" if you use a4 paper size
    %toggle to 'draft' version in order to save compilation time
    %\documentclass[a4,12pt,draft]{article}
    \documentclass[12pt]{article}
  
% TITLE SECTION
	
 %Abstract
    \usepackage{abstract} % Allows abstract customization
    % Set the "Abstract" text to bold
    \renewcommand{\abstractnamefont}{\normalfont\bfseries}
    % Set the abstract itself to small italic text
    \renewcommand{\abstracttextfont}{\normalfont\small\itshape} 

 %Title
    %\usepackage{titlesec} % Allows customization of titles

 %Authors
    \usepackage{authblk} % For multiple authors

 %Date
	\usepackage{datetime} % allows for including today's date
  	% These two lines creates a new date format ``Month day(th), year''
    \newdateformat{usvardate}{
  	\monthname[\THEMONTH] \ordinal{DAY}, \THEYEAR}

% HEADERS & FOOTERS

 %Footnotes
  \usepackage[bottom]{footmisc} % Makes footnotes stick to bottom of the page

 %Endnotes
	% Uncomment this line if using endnotes "\endnote{}"
	% \usepackage{endnotes}
    
 %Headers from page 2 on
%     \usepackage{fancyhdr}
%     \pagestyle{fancy}
%     \fancyheadoffset{0cm}
%     \setlength{\headheight}{15pt} 
 
% DRAFT WATERMARK
	% NOTE: Comment out these two lines to remove watermark
    % \usepackage[firstpage]{draftwatermark} % adds draft watermark
    % \SetWatermarkLightness{0.85}

% MACROS
    % Define keywords macro command
    \providecommand{\keywords}[1]{\textbf{\textit{Keywords---}} #1}

	% Word Count Macro (REQUIRES: xeletax compilation)
    % See NOTE in README.txt for instructions
    %\usepackage{xesearch}
    \newcounter{words}
    \newenvironment{wordcount}{%
      \setcounter{words}{0}
      \SearchList!{wordcount}{\stepcounter{words}}
        {a?,b?,c?,d?,e?,f?,g?,h?,i?,j?,k?,l?,m?,
        n?,o?,p?,q?,r?,s?,t?,u?,v?,w?,x?,y?,z?}
      \UndoBoundary{'}
      \SearchOrder{p;}}{%
      \StopSearching}

% MATH SUPPORT
    % The amssymb package provides various useful mathematical symbols
    \usepackage{amssymb}
    % The amsthm package provides extended theorem environments
    \usepackage{amsthm}
    % The newtxmath package provides additional math symbol support
    	% in Times New Roman symbols, etc.
    %\usepackage{newtxmath}
    % The amsmath package provides miscellaneous enhancements for improving the information structure
    \usepackage{amsmath}

% FONTS
    \usepackage{microtype} % Slightly tweak font spacing for aesthetics
    \usepackage[utf8]{inputenc}
    %\usepackage{newtxtext} % Makes default font Adobe Times New Roman
  
% LINES
	% Spacing
	\usepackage{setspace} % See \doublespacing command at the top of content.tex
    % Numbering
    \usepackage{lineno} 	% See \linenumbers at the top of content.tex
    \usepackage[table]{xcolor}

% MARGINS
	%NOTE: All spaces in this template are in inches, because it is
    % formatted for letterpaper (8.5 x 11 inch) paper. If you use a4
    % paper, choose different sizes in millimeters or centimeters.
    %margin space for comments
    \usepackage[top=1.5cm, bottom=1.5cm, outer=3.5cm, inner=2cm, heightrounded, marginparwidth=2.5cm, marginparsep=0.75cm]{geometry}
    %final print:
    %\usepackage[top=1.5cm, bottom=1.5cm, outer=2cm, inner=2cm, heightrounded, marginparwidth=1.5cm, marginparsep=1.0cm]{geometry}

% COMMENTS
    %\usepackage{marginnote}
    \usepackage[colorinlistoftodos]{todonotes} % allows margin comments
    % See examples in content.tex, and here for manual: 
    % http://www.ctan.org/pkg/todonotes
    \usepackage{soul} % allows for highlighting
    
% GRAPHICS
    \usepackage{graphicx} % More advanced figure inclusion
    \usepackage{float} % For specifying table/figure locations, i.e. [ht!]
    \usepackage{array} % For having tables with paragraph-style columns aligned in middle
    \usepackage{rotating} % For enabling sideway figures (i.e. landscape)
    \usepackage{subcaption} %For enabling subfigures
    % The printlen command allows the user to print the exact text width or height.
    % This is useful, when trying to create graphics (outside of LaTeX, of course)
    % with the optimal dimensions. See here for usage: http://www.ctan.org/pkg/printlen
    \usepackage{printlen}
    \usepackage[percent]{overpic}

% TABLES
    %\usepackage{longtable} % For long tables that span multiple pages
    %\usepackage{tabularx} % Allows advanced table features
    \usepackage{colortbl,textcomp,cancel}%%,setspace
    \PassOptionsToPackage{table}{xcolor}
    %\usepackage{xr} 
    \newcolumntype{L}[1]{>{\raggedright\arraybackslash}p{#1}}
    \newcolumntype{C}[1]{>{\centering\arraybackslash}p{#1}}
    \newcolumntype{R}[1]{>{\raggedleft\arraybackslash}p{#1}}
    \usepackage{relsize} % Allows precise adjustment of font size,
    	%useful for fitting tables to page width
        
% UNITS
    \usepackage{units}

% STRUCTURE
    \usepackage{placeins}
    \usepackage{enumitem}
    
% REFERENCES
	\usepackage{hyperref} % For hyperlinks in the PDF

      % NOTE: This document uses bibtex for references
      % because both style files for Sociology Journals are
      % only available in .bst format. You can change to
      % biblatex if you prefer below.

      % Many thanks to Sociology Professor, 
      % ChangHwan Kim at the University of Kansas. 
      % for providing these .bst files here: 
      % http://people.ku.edu/~chkim
      
	% BIBTEX
    % Comment out this line if using biblatex
    %\usepackage{chicago} % AJS and ASR styles rely on chicago
    \usepackage[authoryear]{natbib}
    
    % BIBLATEX
    % NOTE: Uncomment out these three lines to use biblatex
    % Be sure to put the biblio.bib file in biblatex format'
    %\usepackage{csquotes}
	%\usepackage[style=authoryear,backend=biber]{biblatex}
	%\bibliography{biblio}

\pagenumbering{gobble}% Remove page numbers (and reset to 1)

\graphicspath{
%{./figures/}
%{./figures_prep/}
{/home/onur/setups/test-BGCmodels/nflexpd/1D-ideal-highlat-MS/21-07-29/}
%{/home/onur/setups/test-BGCmodels/nflexpd/1D-ideal-highlat/20-10-12/FS_fC_const/}
%{/home/onur/setups/test-BGCmodels/nflexpd/1D-ideal-highlat/20-10-12/FS_fC_LN/}
}

\newcommand{\done}{\textcolor{green}{done}}
\newcommand{\fail}{\textcolor{red}{FAIL!!}}

\begin{document}

%\input{figs_fsingle}
%Fig.1
\begin{figure}[ht!]
\includegraphics[width=12cm,trim=0mm 54mm 0cm 0mm, clip]{figures/2022-03-22a_Ld1_T20_ThatVar_PARint_vs_PARext/0D-Highlat_wconst_lext_Ld1_T20_CbasedIA_modular_24h_vs_0D-Highlat_wconst_lint_Ld1_T20_CbasedIA_modular_24h_cont_abio0_1y.png}
%\caption{Daily average irradiance and its temporal derivative, as extracted from the simulation outputs generated for T1. PAR:N (solid blue line) is the model version where the temporal derivative of irradiance is approximated numerically; PAR:A (dashed orange line): both irradiance and its temporal derivative are calculated analytically.\label{f.T1light}}
\end{figure}

\clearpage
%Fig.2
\begin{figure}[ht!]
\includegraphics[width=12cm,trim=0cm 16.8cm 0.0cm 5mm, clip]{figures/2022-03-22a_Ld1_T20_ThatVar_PARint_vs_PARext/0D-Highlat_wconst_lext_Ld1_T20_CbasedIA_modular_24h_vs_0D-Highlat_wconst_lint_Ld1_T20_CbasedIA_modular_24h_cont_abio1_1y.png}
%\caption{T1: Carbon (left) and nitrogen (right) pools for the PAR:N (solid blue line), and PAR:A (dashed orange line) simulations. \label{f.T1res}}
\end{figure}

\clearpage
%Fig.3
\begin{figure}[ht!]
  \includegraphics[width=12cm,trim=0mm 0mm 0cm 0mm, clip]{figures/2022-05-06a_Ldvar_Tvar_ThatVar_PARext_IA_vs_DA/0D-Highlat_wconst_lext_Ldvar_Tvar_DIN015_CbasedIA_modular_24h_vs_0D-Highlat_wconst_lext_Ldvar_Tvar_DIN015_CbasedDA_modular_24h_cont_abio0_2y.png}
  %\caption{Daily average irradiance and temperature and their numerically approximated temporal derivatives used in T2.\label{f.T2env}}
\end{figure}

\clearpage
%Fig.4
\begin{figure}[htb!]
\includegraphics[width=14cm,trim=0cm 11mm 0.0cm 5mm, clip]{figures/2022-05-06a_Ldvar_Tvar_ThatVar_PARext_IA_vs_DA/0D-Highlat_wconst_lext_Ldvar_Tvar_DIN015_CbasedIA_modular_24h_vs_0D-Highlat_wconst_lext_Ldvar_Tvar_DIN015_CbasedDA_modular_24h_cont_abio1_2y.png}
%\caption{T2: Carbon (left) and nitrogen (right) pools for the IA (solid blue line) and DA (dashed orange line) variants with variable daylength and temperature. \label{f.T2res}}
\end{figure}

\clearpage
%Fig.5
\begin{figure}[htb!]
\includegraphics[width=14cm,trim=0cm 11mm 0.0cm 5mm, clip]{figures/2022-05-06a_Ldvar_Tvar_ThatVar_PARext_IA_vs_DA/IA-DIF-DA_cont_abio1_1y.png}
%\caption{T2: Differences between the IA and DA variants for the quantities shown in Fig.~\ref{f.T2res}.\label{f.T2resdif}}
\end{figure}

\clearpage
%Fig.6
\begin{figure}[ht!]
\includegraphics[width=16cm,trim=0cm 0mm 0.0cm 0mm, clip]{figures/2022-05-07_Ldvar_Tvar_ThatVar_10P_PARext_IA_vs_DA/0D-Highlat_wconst_lext_Ldvar_Tvar_DIN015_10P_Cbased_IAvsDAvsDIF_lineplot_size_totphy.png}
%\caption{T3: Top row: C biomass of 10 phytoplankton size classes in the IA (solid line) and DA (dotted line) variants (left) and the difference between IA and DA (right). Bottom row: sums of phytoplankton C biomass simulated by the IA and DA variants (left) and their differences (right).}\label{f.T3res}
\end{figure}

\clearpage
%Fig.7
\begin{figure}[ht!]
\includegraphics[width=12cm,trim=0cm 18mm 0.0cm 15mm, clip]{figures/2022-05-06c_Ldvar_Tvar_ThatVar_PARext_Dvar_sdet01_IA_vs_DA/0D-Highlat_wconst_lext_Ldvar_Tvar_DIN015_2P_Dmax1_sdet01_CbasedIA_modular_24h_vs_0D-Highlat_wconst_lext_Ldvar_Tvar_DIN015_2P_Dmax1_sdet01_CbasedDA_modular_24h_cont_abio1_2y.png}
%\caption{T4: Annual variations of global C and N ($\text{total C} + \text{Ext}_{\text{C}}$ and $\text{total N} + \text{Ext}_{\text{N}}$, see Eq.~\ref{eq:Xext}), total C and N ($\sum\text{Phy}_{\text{C}}^j + \text{DIC} + \text{DOC} + \text{Det}_{\text{C}}$ and $\sum\text{Phy}_{\text{N}}^j + \text{DIN} + \text{DON} + \text{Det}_{\text{N}}$) and other state variables that trace individual C and N pools for a seasonally varying mixing regime in an open system.\label{f.T4res}}
\end{figure}

\clearpage
%Fig.8
\begin{figure}[ht!]
\includegraphics[width=12cm,trim=0cm 0mm 0.0cm 0mm, clip]{figures/2022-05-06a_Ldvar_Tvar_ThatVar_PARext_IA_vs_DA/0D-Highlat_wconst_lext_Ldvar_Tvar_DIN015_CbasedIA_modular_24h_vs_0D-Highlat_wconst_lext_Ldvar_Tvar_DIN015_CbasedDA_modular_24h_cont_phy-4_2y.png}
%\caption{T2: Total $\textrm{d}Q/\textrm{d}t$ and its components as contributed by the changes in DIN, $\bar{I}$, $\textrm{L}_{\textrm{D}}$ and T.\label{f.T2dQdt}}
\end{figure}


\end{document}
