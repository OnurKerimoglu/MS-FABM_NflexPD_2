%% Copernicus Publications Manuscript Preparation Template for LaTeX Submissions
%% ---------------------------------
%% This template should be used for copernicus.cls
%% The class file and some style files are bundled in the Copernicus Latex Package, which can be downloaded from the different journal webpages.
%% For further assistance please contact Copernicus Publications at: production@copernicus.org
%% https://publications.copernicus.org/for_authors/manuscript_preparation.html


%% Please use the following documentclass and journal abbreviations for preprints and final revised papers.

%% 2-column papers and preprints
%\documentclass[gmd, manuscript]{copernicus}
\documentclass[gmd, manuscript]{copernicus}

%% Journal abbreviations (please use the same for preprints and final revised papers)

% Advances in Geosciences (adgeo)
% Advances in Radio Science (ars)
% Advances in Science and Research (asr)
% Advances in Statistical Climatology, Meteorology and Oceanography (ascmo)
% Annales Geophysicae (angeo)
% Archives Animal Breeding (aab)
% ASTRA Proceedings (ap)
% Atmospheric Chemistry and Physics (acp)
% Atmospheric Measurement Techniques (amt)
% Biogeosciences (bg)
% Climate of the Past (cp)
% DEUQUA Special Publications (deuquasp)
% Drinking Water Engineering and Science (dwes)
% Earth Surface Dynamics (esurf)
% Earth System Dynamics (esd)
% Earth System Science Data (essd)
% E&G Quaternary Science Journal (egqsj)
% European Journal of Mineralogy (ejm)
% Fossil Record (fr)
% Geochronology (gchron)
% Geographica Helvetica (gh)
% Geoscience Communication (gc)
% Geoscientific Instrumentation, Methods and Data Systems (gi)
% Geoscientific Model Development (gmd)
% History of Geo- and Space Sciences (hgss)
% Hydrology and Earth System Sciences (hess)
% Journal of Bone and Joint Infection (jbji)
% Journal of Micropalaeontology (jm)
% Journal of Sensors and Sensor Systems (jsss)
% Magnetic Resonance (mr)
% Mechanical Sciences (ms)
% Natural Hazards and Earth System Sciences (nhess)
% Nonlinear Processes in Geophysics (npg)
% Ocean Science (os)
% Primate Biology (pb)
% Proceedings of the International Association of Hydrological Sciences (piahs)
% Scientific Drilling (sd)
% SOIL (soil)
% Solid Earth (se)
% The Cryosphere (tc)
% Weather and Climate Dynamics (wcd)
% Web Ecology (we)
% Wind Energy Science (wes)


%% \usepackage commands included in the copernicus.cls:
%\usepackage[german, english]{babel}
%\usepackage{tabularx}
%\usepackage{cancel}
%\usepackage{multirow}
%\usepackage{supertabular}
%\usepackage{algorithmic}
%\usepackage{algorithm}
%\usepackage{amsthm}
%\usepackage{float}
%\usepackage{subfig}
%\usepackage{rotating}

%Additional packages: 
\usepackage{mathrsfs} %provides calligraphic Fonts
\usepackage{placeins}
\usepackage[percent]{overpic}
\allowdisplaybreaks

\graphicspath{
%{./figures/common/}
%{./figures/v_10_13_solitary/}
{./figures/}
}

\newcommand{\onur}[1]{\textcolor{blue}{\{Onur: #1\}}}
\newcommand{\markus}[1]{\textcolor{blue}{\{Markus: #1\}}}

\begin{document}
\title{FABM-NflexPD 2.0: Testing an Instantaneous Acclimation APproach for Modelling the Implications of Phytoplankton Eco-physiology on the carbon and nutrient cycles}

% \Author[affil]{given_name}{surname}

\Author[1]{Onur}{Kerimoglu}
\Author[2]{Markus}{Pahlow}
\Author[3,a]{Prima}{Anugerahanti}
\Author[3]{S. Lan}{Smith}

\affil[1]{Institute for Chemistry and Biology of the Marine Environment, University of Oldenburg, Oldenburg, Germany}
\affil[2]{GEOMAR Helmholtz Centre for Ocean Research Kiel, Kiel, Germany}
\affil[3]{Earth SURFACE Research Center, Research Institute for Global Change, JAMSTEC, Yokosuka, Japan}
\affil[a]{Present address: Dept.\ of Earth, Ocean and Ecological Sciences, School of Environmental Sciences, University of Liverpool, Liverpool, United Kingdom}

%% The [] brackets identify the author with the corresponding affiliation. 1, 2, 3, etc. should be inserted.

%% If an author is deceased, please mark the respective author name(s) with a dagger, e.g. "\Author[2,$\dag$]{Anton}{Aman}", and add a further "\affil[$\dag$]{deceased, 1 July 2019}".

%% If authors contributed equally, please mark the respective author names with an asterisk, e.g. "\Author[2,*]{Anton}{Aman}" and "\Author[3,*]{Bradley}{Bman}" and add a further affiliation: "\affil[*]{These authors contributed equally to this work.}".


\correspondence{Onur Kerimoglu (kerimoglu.o@gmail.com)}

\runningtitle{FABM-NflexPD 2.0}

\runningauthor{Kerimoglu et al.}


\received{}
\pubdiscuss{} %% only important for two-stage journals
\revised{}
\accepted{}
\published{}

%% These dates will be inserted by Copernicus Publications during the typesetting process.


\firstpage{1}

\maketitle


\begin{abstract}
The acclimative response of phytoplankton, which adjusts their nutrient and pigment content in response to changes in ambient light, nutrient levels, and temperature, is well appreciated as an important determinant of observed chlorophyll distributions and biogeochemistry. Various acclimative models have been developed to capture this response and its impact on the C:nutrient:Chl ratios of phytoplankton, and thereby on the cycling of carbon and nutrients. Compared to typical acclimative models that explicitly resolve (via one differential equation for each) the different nutrients or elements in phytoplankton biomass, the Instantaneous Acclimation (IA) approach has recently been shown to capture this acclimative response with greater computational efficiency, without substantial loss of accuracy in both 0D box models and spatially explicit 1D models. IA allows calculating the full dynamics of only one tracer (in terms of either carbon or a single nutrient) of phytoplankton biomass, while calculating the remaining ratios assuming instantaneous local equilibrium within each modelled grid cell, at each time step. We have recently extended this IA model to capture both C and N cycles, and conserve  both total C and N in a 0D setup. We present extensive tests of this model, compared to an otherwise equivalent fully dynamic model, under idealized conditions with variable light and temperature. We also demonstrate a new module implementing this model in the Framework for Aquatic Biogeochemical Modelling (FABM), which facilitates modelling competition between an arbitrary number of different acclimative phytoplankton types. This provides a new tool for exploring the links between phytoplankton eco-physiology, ecological dynamics, and aquatic biogeochemistry. 
\end{abstract}


%\copyrightstatement{TEXT}

%INTRODUCTION
\introduction%% \introduction[modified heading if necessary]

Elemental stoichiometry and pigment density of phytoplankton exhibit strong variability across environmental conditions, both at a physiological level \citep[e.g.][]{Garcia2016}, and at a community level \citep[e.g.][]{MorenoMartiny2018}. The physiological flexibility is recognized to be driven by an acclimative re-adjusment of cellular machinery to changes in the availability in nutrient and light, and the fact that the cellular apparatus involved in various cellular functions have different elemental compositions, an example being chlorophyll (hereafter Chl) pigments being rich in N \citep{Geider2002}. The systematic relationships observed between species and cellular compositions is explained by the typical composition of some species being better or worse equipped than those of others for a given resource regime \citep{Arrigo2005,Burson2016}.

Such variabilities in the cellular composition of phytoplankton being potentially relevant for biogeochemical cycles have been recognized decades ago \citet{Redfield1934,Redfield1958}, and evidence have been building since then \citep{Lenton2007,Bonachela2016,Pahlow2020}. From a modelling point of view, it has been shown that accounting for acclimative capacity of phytoplankton is relevant for predicting the response of ecosystems to environmental change \citep{Kwiatkowski2018,Kerimoglu2018}, for model performance \citep{Ayata2013,Kerimoglu2017,Chen2018a}, and endows models with desirable properties such as reduced sensitivity to prescribed parameter values \citep{Anugerahanti2021}. However, such mechanistic models typically require additional state variables, typically one for chloropyll content, and one for each of the resolved nutrients \citep[e.g.][]{Geider1998,Flynn2003}, but possibly even more \citep{Bonachela2013,Wirtz2016,Inomura2020} \onur{Check again Inomura2020}. 

However, additional state variables can increase the computational costs significantly in spatially explicit setups \onur{ANY GOOD REF HERE?}, especially for models that describe multiple dozens or hundreds of phytoplanton groups \citep[e.g.][]{Dutkiewicz2020}. As a potential remedy to this problem, the `instantaneous acclimation' approach can be used, where the changes in cellular composition are not dynamically tracked, but are assumed to instantaneously adjust to the resource environment. For instance, \citet{Pahlowetal13} described a model where the Chl:C ratio is instantaneously adjusted to any given irradiance level, based on a trade-off between the higher specific production rate benefit of a higher Chl:C ratio and the additional respiratory costs of synthesizing and maintaining the pigments. This was followed by the model by \citet{Smith2016}, where an additional step was taken to assume that the C:N ratio is adjusted instantaneously, to the equilibrium ratio to be reached under balanced growth \citep{Burmaster1979}. Following up the idea, \citet{Ward2017} evaluated the accuracy of this approach by comparing a fully explicit, classical Caperon/Droop \citep{Caperon1968,Droop1968} model to its instantaneous counterpart, and using a 0D setup, found that across a range of environmental settings, the predictions of the two approaches matched. Recently, \citet{Kerimoglu2021} implemented the `NflexPD' model of \citet{Smith2016} in the Framework of Aquatic Biogeochemical Models \citet[FABM][]{Bruggeman2014}, and using a 1D setup, showed that the predictions of the 'Instantaneous Acclimation' variant of the model largely matched those of the fully explicit 'Dynamic Acclimation' counterpart, except during the transitions between winter-spring and autumn-winter.

The version of the model presented by \citet{Kerimoglu2021} can only resolve N in the Dissolved Inorganic Material (DIM) pool, which may be sufficient for some ecological applications, but not for applications that require resolution of multiple nutrients and carbon in DIM form. Here we introduce FABM-NflexPD 2.0, which can resolve both C, and N DIM pool. We present a detailed description of a C-based version, which is extended to trace DIC, and the N fluxes occurring due to instantaneous changes in cell quotas, such that C and N cycles are both closed. For evaluating the consistency and robustness of the model, we present following formal tests:
\begin{description}
 \item [T1] an assessment of the conservation of the C and and N by a simplified version of the model in a 0D setup, where temperature and day length were held constant, and light and temporal changes in light were provided either externally and numerically approximated respectively, and both calculated analytically.
 \item [T2a] testing the IA approach in a more realistic setup, where irradiance is supplied externally, temperature and day length are allowed to vary, and attenuation of light by phytoplankton is accounted for.
 \item [T2b] comparison of the fully explicit version (the `dynamic acclimation' (DA) variant in K21) vs the instantaneously acclimating (IA) variant of the model 
 \item [T3] demonstration of a model run with multiple phytoplanton groups.
 %\item [T4] \onur{done} application of the model in an idealistic 1D setup and assessment of mass conservation \textcolor{red}{\{Onur: this test failed: in 1D, model does not seem to be mass conservative, see Section~\ref{s.resT4}.\}} 
\end{description}


% \begin{figure}[htb]
%   \centering
%   \includegraphics[width=12cm]{?.pdf}
%   \caption{\label{?}}
% \end{figure}

%MODEL DESCRIPTION
\section{Model Description}
In this section, we will highlight the differences and similarities between the present, `C-based' version of the model from the earlier version of the `N-based' model version presented by K21.

\subsection{C and N content of Phytoplankton}
In this version of the model, C content of phytoplankton ($\text{Phy}_{\text{C}}$) is dynamically tracked (thus, C-based) instead of the N content ($\text{Phy}_{\text{N}}$) as was the case in the earlier (thus, N-based) version. Wherever required, $\text{Phy}_{\text{N}}$ is calculated as:
\begin{equation} \label{eq:Q}
  \text{Phy}_{\text{N}} = \text{Phy}_{\text{C}} \cdot Q_{\text{N}}
\end{equation}

$Q_{\text{N}}$ is assumed to instantaneously adjust to a balanced-growth optimum value, as determined by the nutrient uptake in the protoplast, $\hat{V}_{\text{N}}$ and net photosynthesis in the chloroplast, $\hat{\mu}_{\text{net}}$ (Eq.~10 in K21):
\begin{equation}\label{eq:Qopt}
 Q_{\text{N}}^o= \frac{Q_{\text{N,0}}}{2} \left[1+\sqrt{1+\frac{2}{Q_{\text{N,0}}{({\hat{\mu}_{\text{net}}}/{\hat{V}_{\text{N}}}+\zeta_{\text{N}} )}}} \right] \hfill \{\text{IA}\}
\end{equation}
where $Q_{\text{N,0}}$ and $\zeta_{\text{N}}$ are model parameters (subsistence N quota, and cost of N uptake, respectively, see Table~3 in K21). The rate of change of $\text{Phy}_{\text{C}}$ is given by:
\begin{equation} \label{eq:sPhyC}
\frac{\mathrm{d}\, \text{Phy}_{\text{C}}}{\mathrm{d}\, t} = F_{\text{DIC}-\text{Phy}_{\text{C}}} - F_{\text{Phy}_{\text{C}}-\text{Det}_{\text{C}}}
\end{equation}
which is identical to the source term of the `DA' variant (Eq.~1b in K21). Here, $F_{\text{DIC}-\text{Phy}_{\text{C}}}$ and $F_{\text{Phy}_{\text{C}}-\text{Det}_{\text{C}}}$ are the fluxes from Dissolved Inorganic Carbon (DIC) to phytoplankton, and from phytoplankton to detrital C (Det$_{\text{C}}$), respectively. The first term in Eq.~\eqref{eq:sPhyC} represents net phytoplankton growth, and as before (Eq.~8 in K21), it is given by the product of Phy$_{\text{C}}$ with net growth rate, $\mu$:
\begin{equation} \label{eq:fdicphyc}
 F_{\text{DIC}-\text{Phy}_{\text{C}}} = \mu \cdot \text{Phy}_{\text{C}}
\end{equation}
This term is used in this updated model version also to track the DIC concentration, which in turn enables tracing the total C in the system:
\begin{equation} \label{eq:dic}
  \frac{\mathrm{d}\, \text{DIC}}{\mathrm{d}\, t} = \underbrace{F_{\text{DOC}-\text{DIC}}}_{\textrm{Remineralization}} - \underbrace{F_{\text{DIC}-\text{Phy}_{\text{C}}}}_{\textrm{Net C uptake}}
\end{equation}
Note that representing the C uptake and respiration terms with a lumped term $F_{\text{DIC}-\text{Phy}_{\text{C}}}$ in Eq.~\eqref{eq:dic} is based on the assumption that the respiration terms (see Eq.~7 in K21) are added back to DIC, and would have to be separated if CO2 was explicitly resolved. The remineralization flux, $F_{\text{DOC}-\text{DIC}}$ is calculated as a first-order reaction as in K21, which was previously only used to trace the DOC (K21).

The second term in Eq.~\eqref{eq:sPhyC}, $F_{\text{Phy}_{\text{C}}-\text{Det}_{\text{C}}}$, representing the mortality of phytoplankton, is calculated as a quadratic rate as in K21 (Table 1), but now it is directly based on $\text{Phy}_{\text{C}}$ concentration, i.e.,
\begin{flalign}\label{eq:mortC}
F_{\text{Phy}_{\text{C}}-\text{Det}_{\text{C}}} &= m_{\text{C}} \cdot \text{Phy}_{\text{C}}^2
\end{flalign}
with $m_{\text{C}}$ [\unit{m^3\ mmolC^{-1}\ d^{-1}}] being the C-based specific mortality rate.  The N counterpart, $F_{\text{Phy}_{\text{N}}-\text{Det}_{\text{N}}}$ is found by multiplying this term by $Q_{\text{N}}$.

%given by the division of N flux from phytoplankton to detritus, $F_{\text{Phy}_{\text{N}}-\text{Det}_{\text{N}}}$ by $Q_{\text{N}}$

\subsection{Flux of nutrients between DIM and Phytoplankton}\label{S:DescFlux}

As in K21, the source term for the nutrients in the DIM pool are described as the remineralization fluxes from the nutrients in dissolved organic form (e.g., DON), $F_{\text{DON}-\text{DIN}}$ (Table 1, K21), minus the fluxes between the DIM (e.g., DIN) and phytoplankton ($F_{\text{DIN}-\text{Phy}_{\text{N}}}$):
\begin{equation} \label{eq:sdin}
  \frac{\text{d}\,\text{DIN}}{\text{d}\,t} = F_{\text{DON}-\text{DIN}} - F_{\text{DIN}-\text{Phy}_{\text{N}}}
\end{equation}
$F_{\text{DON}-\text{DIN}}$ is calculated like the $F_{\text{DOC}-\text{DIC}}$ explained above. $F_{\text{DIN}-\text{Phy}_{\text{N}}}$ is N uptake by phytoplankton ($V^{\text{N}}$), which can be expressed as:
\begin{equation} \label{eq:dphyNdt}
  F_{\text{DIN}-\text{Phy}_{\text{N}}} = V_{\text{N}} = F_{\text{DIC -- Phy}_{\text{C}}} \cdot Q_{\text{N}} = \mu \cdot \text{Phy}_{\text{C}} \cdot Q_{\text{N}}
\end{equation}
$Q_{\text{N}}$ changes over time, which contributes to the rate of change of $\text{Phy}_{\text{N}}$:
\begin{equation}
  \label{eq:phyn}
  \frac{\mathrm{d}\,\text{Phy}_{\text{N}}}{\mathrm{d}\,t} = \frac{\mathrm{d}\,\text{Phy}_{\text{C}}\cdot Q_{\text{N}}}{\mathrm{d}\,t}
  = \frac{\mathrm{d}\,\text{Phy}_{\text{C}}}{\mathrm{d}\,t} \cdot Q_{\text{N}} + \text{Phy}_{\text{C}} \cdot \frac{\mathrm{d}\,Q_{\text{N}}}{\mathrm{d}\,t}
  = V_{\text{N}} - F_{\text{Phy}_{\text{C}}-\text{Det}_{\text{C}}}\cdot Q_{\text{N}} + \text{Phy}_{\text{C}} \cdot \frac{\mathrm{d}\,Q_{\text{N}}}{\mathrm{d}\,t}
\end{equation}
The right-most term in Eq.~\eqref{eq:phyn}, $\text{Phy}_{\text{C}}\cdot \mathrm{d}\,Q_{\text{N}} / \mathrm{d}\, t$, is not accounted for because $\text{Phy}_{\text{N}}$ is not a state variable.  Hence, total N is not conserved.  If $\text{Phy}_{\text{N}}$ was a state variable, this term would not appear and mass balance would be ensured by tracking the temporal evolution of $\text{Phy}_{\text{N}}$.  Owing to the instantaneous acclimation (IA) of $Q_{\text{N}}$, we must explicitly account for this term in order to maintain mass balance for N\@.  It is impossible to do this mechanistically because that would require tracking the corresponding N flux in the (nonexistent) state variable $\text{Phy}_{\text{N}}$.  Instead, we have to account for this term by modifying the rate of change of one or more of the N-containing state variables of our IA model.  As done by \citet{Smith2016}, we subtract it from the differential equation for DIN\@:
\begin{flalign}\label{eq:sdin2}
  \begin{split}
    \frac{\text{d}\,\text{DIN}}{\text{d}\, t} &= F_{\text{DON}-\text{DIN}} - V_{\text{N}} - \text{Phy}_{\text{C}} \cdot \frac{\mathrm{d}\, Q_{\text{N}}}{\mathrm{d}\, t} \\
    &= F_{\text{DON}-\text{DIN}} - V_{\text{N}} - \text{Phy}_{\text{C}} \cdot \left(\frac{\partial Q_{\text{N}}}{\partial \text{DIN}} \frac{\text{d}\, \text{DIN}}{\text{d}\, t}
      + \frac{\partial Q_{\text{N}}}{\partial \bar{I}} \frac{\text{d}\, \bar{I}}{\text{d}\, t}
      + \frac{\partial Q_{\text{N}}}{\partial L_{\text{D}}} \frac{\text{d}\, L_{\text{D}}}{\text{d}\, t}
      + \frac{\partial Q_{\text{N}}}{\partial T} \frac{\text{d}\, T}{\text{d}\, t} \right)
  \end{split}
  \\
  \intertext{and reorganizing:}
%intermediate step:
% \begin{equation}\label{eq:sdin2.5}
%  \frac{\text{d}\text{DIN}}{\text{d}t} \left( 1+\text{Phy}_{\text{C}}\frac{\partial Q_{\text{N}}}{\partial \text{DIN}} \right) = F_{\text{DON}-\text{DIN}} - \text{Phy}_{\text{C}} \left(V_{\text{N}} +  \frac{\partial Q_{\text{N}}}{\partial \bar{I}} \frac{\text{d} \bar{I}}{\text{d} t} \right)\\
% \end{equation}
\label{eq:sdin3}
\frac{\text{d}\,\text{DIN}}{\text{d}\,t} &= \frac{\displaystyle F_{\text{DON}-\text{DIN}} - V_{\text{N}} - \text{Phy}_{\text{C}} \cdot
  \left( \frac{\partial Q_{\text{N}}}{\partial \bar{I}} \frac{\text{d}\, \bar{I}}{\text{d}\, t}
    + \frac{\partial Q_{\text{N}}}{\partial L_{\text{D}}} \frac{\text{d}\, L_{\text{D}}}{\text{d}\, t}
    + \frac{\partial Q_{\text{N}}}{\partial T} \frac{\text{d}\, T}{\text{d}\, t} \right)}%
{\displaystyle 1+ \text{Phy}_{\text{C}}\frac{\partial Q_{\text{N}}}{\partial \text{DIN}}}
\end{flalign}
While this, rather arbitrary, measure for achieving N mass balance violates the assumptions behind the model by assigning part of $\text{Phy}_{\text{N}}$ to DIN, the resulting differences compared to explicitly resolving $\text{Phy}_{\text{N}}$ are relatively small \citep{Ward2017}.

% Here, $\frac{\text{d} \text{Phy}_{\text{C}}}{\text{d} t} \big\rvert_G$ corresponds to gross growth of phytoplankton C biomass, i.e., $F_{\text{DIC}-\text{Phy}_{\text{C}}}$ in Eq.~\eqref{eq:sPhyC}. Substituting therefore Eq.~\eqref{eq:fdicphyc} in Eq.~\eqref{eq:dphyNdt}:
% \begin{equation} \label{eq:dphyNdt2}
%   \frac{\text{d}\text{Phy}_{\text{N}}}{\text{d}t} \bigg\rvert_G= Q_{\text{N}} \cdot \mu \cdot \text{Phy}_{\text{C}} + \text{Phy}_{\text{C}} \frac{\text{d} Q_{\text{N}}}{\text{d} t}
% \end{equation}
% where, assuming balanced growth \citep{Burmaster1979}, i.e., $V_{\text{N}} = Q_N \cdot \mu$  (K21, Eq.~6) the first term on the right hand side of Eq.~\eqref{eq:dphyNdt2} can be replaced with $V_{\text{N}} \cdot \text{Phy}_{\text{C}}$, i.e.,
% \begin{equation} \label{eq:dphyNdt3}
%   \frac{\text{d}\text{Phy}_{\text{N}}}{\text{d}t} \bigg\rvert_G=  \text{Phy}_{\text{C}} \left( V_{\text{N}} + \frac{\text{d} Q_{\text{N}}}{\text{d} t} \right)
% \end{equation}

% %\onur{In the N-based version, we had (K21, Eq.~5): $F_{DIN-%\text{Phy}_{\text{N}}} = V_{\text{N}} \cdot \text{Phy}_{\text{C}}$. i.e., excluding the $%\frac{\text{d} Q_{\text{N}}}{\text{d} t}$ term in Eq.~\eqref{eq:dphyNdt3}. %As we discussed, this apparent difference %(apparent, because maybe there is a mistake in the notation? Also note that $F_{DIN-\text{Phy}_{\text{N}}}$ is not used anywhere, and in the C-based version of the model it is only saved as a diagnostic quantity)
% %doesn't violate the internal consistency of the N-based version, but it would be great if we could explain what this really is. For instance, is this a `real deviation' in model formulations that contributes (among other sources) to quantitative differences in results?}

% According to Eq.~\eqref{eq:dphyNdt3}, the N flux between the DIM pool and phytoplankton can be intuitively understood as the sum of uptake due to growth of C biomass, and the change in the N quota. % It should be noted that this cumulative change can become negative. (?)

% Given that the terms $\hat{V}_{\text{N}}$ and $\hat{\mu}_{\text{net}}$ required for calculating $Q_{\text{N}}$ (Eq.~\ref{eq:Qopt}) are respectively functions of DIN (Eq.~16 in K21) and daytime average irradiance, $\bar{I}$ (Eqs.~20--22 in K21), the total time derivative of $Q_{\text{N}}$ in Eq.~\eqref{eq:dphyNdt3} can be computed as the sum of partial derivatives with respect to DIN and $\bar{I}$ using chain rule:
% \begin{equation} \label{eq:dqdt}
%  \frac{\text{d} Q_{\text{N}}}{\text{d} t} = \frac{\partial Q_{\text{N}}}{\partial \text{DIN}} \frac{\text{d} \text{DIN}}{\text{d} t} +  \frac{\partial Q_{\text{N}}}{\partial \bar{I}} \frac{\text{d} \bar{I}}{\text{d} t}
% \end{equation}
% Substituting Eq.~\eqref{eq:dqdt} in Eq.~\eqref{eq:dphyNdt3}, and subsequently Eq.~\eqref{eq:dphyNdt3} in Eq.~\eqref{eq:sdin}:
% \begin{equation}\label{eq:sdin2}
%  \frac{\text{d}\text{DIN}}{\text{d}t} = F_{\text{DON}-\text{DIN}} - \text{Phy}_{\text{C}} \left(V_{\text{N}} + \frac{\partial Q_{\text{N}}}{\partial \text{DIN}} \frac{\text{d} \text{DIN}}{\text{d} t} +  \frac{\partial Q_{\text{N}}}{\partial \bar{I}} \frac{\text{d} \bar{I}}{\text{d} t} \right)
% \end{equation}
% and reorganizing:
% %intermediate step:
% % \begin{equation}\label{eq:sdin2.5}
% %  \frac{\text{d}\text{DIN}}{\text{d}t} \left( 1+\text{Phy}_{\text{C}}\frac{\partial Q_{\text{N}}}{\partial \text{DIN}} \right) = F_{\text{DON}-\text{DIN}} - \text{Phy}_{\text{C}} \left(V_{\text{N}} +  \frac{\partial Q_{\text{N}}}{\partial \bar{I}} \frac{\text{d} \bar{I}}{\text{d} t} \right)\\
% % \end{equation}
% \begin{equation}\label{eq:sdin3}
%  s(\text{DIN}) = \frac{\text{d}\text{DIN}}{\text{d}t} = \frac{F_{\text{DON}-\text{DIN}} - \text{Phy}_{\text{C}} \left(V_{\text{N}} +  \frac{\partial Q_{\text{N}}}{\partial \bar{I}} \frac{\text{d} \bar{I}}{\text{d} t} \right)}{ 1+\text{Phy}_{\text{C}}\frac{\partial Q_{\text{N}}}{\partial \text{DIN}}}
% \end{equation}
As a technical remark regarding the FABM implementation of the model: Eq.~\eqref{eq:sdin3} requires a combination of terms, which, in K21, used to be calculated by separate abiotic ($F_{\text{DON}-\text{DIN}}$) and phytoplankton modules (all other terms), therefore in the current implementation, in order to avoid a circular-dependency error, an intermediate module collects the necessary terms from the two modules, and sets the right hand sides for DIN at once.

In Eq.~\eqref{eq:sdin3}, partial derivatives of $Q_{\text{N}}$ with respect to DIN, $\bar{I}$, $L_{\text{D}}$ and $T$ are obtained by canonical application of the chain rule, as detailed in Appendix~\ref{S:Sol}.  

The final terms required in Eq.~\eqref{eq:sdin3} are the changes in $\bar{I}$, $L_{\text{D}}$ and $T$ over time, i.e., $\text{d}\,\bar{I} / \text{d}\,t$, $\text{d}\,L_{\text{D}} / \text{d}\,t$
and $\text{d}\,T / \text{d}\,t$.
When the irradiance and temperature are supplied externally, as typically the case in realistic coupled physical-biological models it is not possible to obtain their temporal derivative analytically, therefore they are numerically approximated as the discrete backward difference between its current ($t=i$) and previous ($t=i-1$) values, divided by the size of the integration time step i.e., $\text{d}\, E / \text{d}\, t \approx (E_{i} - E_{i-1}) / \Delta t$ for $E=\{\bar{I},\ T\}$ (see also Section~\ref{S:DescT1} designed to test the sensitivity of the model results and conservation behavior to this approximation for $E=\bar{I}$).

% Table
\begin{table*}[htb]
  \caption{ Descriptions, values and units of model parameters regarding phytoplankton growth. The parameters with prime ($C'$) are given for a cell with an ESD of 1$\mu$m, which is the size assumed for experiments T1-2. For T3, for which different size classes are simulated, the respective values are obtained according to $C=C'\exp{(S_C \log (ESD))}$, where $S_C$ is the allometric scaling coefficient for this paramater. Values for $C'$ and $S_C$ are as in \citet{Smith2016}, and rest of the parameters as in \citet{Kerimoglu2021}. \label{T.pars}}
  \begin{tabular}{l l c l}
    \tophline
    Term/Symbol         & Definition                        & Value     & Unit\\
    \middlehline
    $\hat{\mu}_0$   & Potential maximum growth rate         & 5.0       & \unit{d^{-1}}\\
    $Q_0'$  & Subsistence quota                     & 0.039     & \unit{mmolN\ molC^{-1}}\\
    $S_{Q_0}$  & Allometric scaling coefficient of $Q_0$       & -0.18     & \unit{mmolN\ molC^{-1}}\\
    $\hat{A}_0'$     & Potential maximum nutrient affinity   & 0.15       & \unit{m^3\ mmolC^{-1}\ d^{-1}}\\
    $S_{A_0}$  & Allometric scaling coefficient of $\hat{A}_0$        & -0.8     & \unit{mmolN\ molC^{-1}}\\
    $\hat{V}_0'$     & Potential maximum N uptake rate       & 5.0       & \unit{molN\ molC^{-1}\ d^{-1}}\\
    $S_{V_0}$  & Allometric scaling coefficient of $\hat{V}_0$        & 0.2     & \unit{mmolN\ molC^{-1}}\\
    $\alpha$           & Chl-specific slope of P-I curve    & 1.0       & \unit{m^2\ E\ molC gChl^{-1}\ d^{-1}}\\
    $R^{\text{Chl}}_{\text{M}}$     & Cost of Chl maintenance   & 0.1       & \unit{d^{-1}}\\
    $\zeta_{\text{Chl}}$   & Cost of Chl synthesis  & 0.5       & \unit{mmolC\ gChl^{-1}}\\
    $\zeta_{\text{N}}$     & Cost of N uptake               & 0.6       & \unit{molC\ molN^{-1}}\\
    $m$                & Mortality rate coefficient        & 0.1   & \unit{m^{3} mmolN^{-1} d^{-1}} \\
    $r_{\text{hyd}}$   & Hydrolysis rate constant          & 0.1   & \unit{d^{-1}} \\
    $r_{\text{rem}}$   & Remineralization rate constant    & 0.1   & \unit{d^{-1}} \\
    \bottomhline
  \end{tabular}
  \belowtable{} % Table Footnotes
\end{table*}

\subsection{Test setups and model operation}\label{S:DescSetup}
    %\subsubsection{0D setup (T1-T3)}
    For all tests, the model is operated in a spatially homogeneous `box' setup, using the 0d driver of FABM \citep{Bruggeman2014}. With this 0D setup, numerical solutions are obtained using 4$^\text{th}$ order Runge-Kutta method with a time step of 60 seconds. Model forcing applied in 0D setups vary among different tests, as explained below.

    \subsubsection{T1}\label{S:DescT1}
    As simplifications for this test, we assume that temperature, T is fixed at 10$\degree$C, fractional day length, $L_D$ is unity, and ignore light attenuation and albedo effects. We consider two cases with regard to the handling of irradiance and its temporal derivative:
    \begin{description}
    \item [PAR:N] in the first case, short wave radiation is supplied externally. For being able to compare this approach with the analytical approach described below, the externally supplied photosynthetically available irradiance is calculated as a sinusoidal function:
    \begin{flalign}\label{eq.I}
    \bar{I}(t) &= \bar{I}_{\min} + (\bar{I}_{\max}-\bar{I}_{\min}) 0.5 \left[ 1 + \sin \left[ 2 \pi t' \right]  \right], \qquad t' = \frac{t}{365} - 0.25
    \end{flalign}
    where, $\bar{I}_{\min}= 1.6\, \mathrm{mol\,m^{-2}d^{-1}}$  and $\bar{I}_{\max} = \mathrm{110 \, mol\, m^{-2} d^{-1}}$ define the minimum and maximum values throughout the year, $t$ is the day of the year, and $t'$ represents the relative day of the year delayed by a quarter cycle to obtain the peak value at the middle of the year to represent a seasonal cycle representative of a high latitude environment in the northern hemisphere (see Fig.~\ref{f.T1light} for the behavior of the function with these parameters).
    \item [PAR:A] with the numerical approach described above, approximating the derivative with a discrete difference term can lead to inaccuracies. In order to diagnose such a potential source of error, we consider a a second case, where the temporal derivative of short wave radiation is calculated  analytically:
    \begin{flalign}\label{eq.dIdt}
    \frac{\text{d}\,\bar{I}}{\text{d}\,t} &= (\bar{I}_{\max}-\bar{I}_{\min}) \frac{\pi}{365} \cos \left[ 2 \pi t' \right]
    \end{flalign}
    \end{description}
    The temporal derivatives found by the numerical and analytical approaches are almost identical, as was targeted (Fig.~\ref{f.T1light}).
    
    \begin{figure}[ht!]
    \includegraphics[width=12cm,trim=0mm 54mm 0cm 0mm, clip]{figures/2022-03-22a_Ld1_T20_ThatVar_PARint_vs_PARext/0D-Highlat_wconst_lext_Ld1_T20_CbasedIA_modular_24h_vs_0D-Highlat_wconst_lint_Ld1_T20_CbasedIA_modular_24h_cont_abio0_1y.png}
    \caption{Daily average irradiance and its temporal derivative, as extracted from the simulation outputs generated for T1. PAR:N (solid blue line) is the model version where the temporal derivative of irradiance is approximated numerically; PAR:A (dashed orange line): both irradiance and its temporal derivative are calculated analytically.\label{f.T1light}}
    \end{figure}

    \subsubsection{T2-T3}
    For these tests, we consider variable day length, $L_D$ as described by \citet{Forsythe2003}, which was used to calculate the daytime average irradiance, based on the externally provided incoming irradiance as described for T1 (i.e., $\bar{I}=\bar{I}_{\text{24h}}/L_D$ as in the original model by \citep{Pahlowetal13}, and as was implemented in K21). Moreover, to account for the seasonal variability in T, it was represented using a sinusoidal function similar to Eq.~\ref{eq.I}, but $T_{\min}$ and $T_{\max}$ set to 2 and 20$\degree$C, respectively.
    
    For T3, we simulated 10 phytoplankton groups with the IA and DA variants and compared their predictions. Phytoplankton groups were assumed to represent different size classes spread across 0.2-100 $\mu$m equivalent spherical diameter (ESD) range, uniformly spaced on logarithmic scale. As in \citet{Smith2016}, we assumed 
    $Q_0$, $\hat{V}_0$ and $\hat{A}_0$ to vary based on allometric relationships  (Table~\ref{T.pars}). Scaling of $Q_0$ and $\hat{V}_0$ are based on a combination of cell-specific scaling of subsistence quotas \citep[][`marine species']{Edwards2012} and maximum uptake rates \citep{Maranon2013}, and cell-specific scaling of C content \citep[][`protist plankton excluding diatoms']{Menden2000}. Scaling of $\hat{A}_0$ is based on heuristically, as explained in \citet{Smith2014a}. A detailed discussion of the allometric scaling of parameters can be found in \citet{Smith2016}.
    
%\subsubsection{1D setup (T4)}
%    \onur{I'm not sure if we really need to show these results, although I included them for now }

%\begin{figure}[t]
%\includegraphics[width=8.3cm,trim=0cm 6mm 0.0cm 10mm, clip]{figures/fI_analytical.png}
%\caption{Seasonal course of analytically described (Eqs.\eqref{f.Ian}) daily average irradiance (a) and its temporal derivative (b), with $\bar{I}_{\min}$=1.6 and $\bar{I}_{\max}$=110 [mol m$^2$ d$^{-1}$]. \label{f.Ian}}
%\end{figure}

%and $\frac{\text{d} \text{DIN}}{\text{d} t} \approx \frac{\text{DIN}_{i} - \text{DIN}_{i-1}}{\Delta t}$.

%\newpage

% RESULTS
\section{Results}

\subsection{Accuracy of numerical approximation of temporal derivative of light (T1)}
As shown in Fig.~\ref{f.T1res}, the model is conservative with respect to C and, to a great extent N\@. Total-N as estimated with the numerical approach (PAR:N) slightly (about 0.02 \unit{mmolN\ m^3} increases at the very beginning of the simulation, which is caused by the unknown previous value of $\bar{I}$ during the first day of the simulation (as required by the numerical approximation of $\text{d}\,\bar{I}/\text{d}\,t$), which is set to the current value (such that $\text{d}\,\bar{I}/\text{d}\,t$=0.0), causing therefore a difference between the total N as simulated by the analytical approach (PAR:A). However, after this initial jump, this new baseline is preserved until the end of the simulation.

\begin{figure}[ht!]
\includegraphics[width=12cm,trim=0cm 16.8cm 0.0cm 5mm, clip]{figures/2022-03-22a_Ld1_T20_ThatVar_PARint_vs_PARext/0D-Highlat_wconst_lext_Ld1_T20_CbasedIA_modular_24h_vs_0D-Highlat_wconst_lint_Ld1_T20_CbasedIA_modular_24h_cont_abio1_1y.png}
\caption{Results of T1: Carbon (left) and nitrogen (right) pools according to the PAR:N (solid blue line), and PAR:A (dashed orange line) approaches \onur{I think the other panels (din,phy,det,don) are not necessary to include here}.\label{f.T1res}}
\end{figure}

\FloatBarrier

\subsection{Testing the IA approach in a more realistic setup, in comparison to the fully explicit DA approach (T2)}

For this test, seasonal variations in $T$ and $L_{\text{D}}$ were considered, as shown in (Fig.~\ref{f.T2env}). It should be noted that, because of the variations in $L_{\text{D}}$, the seasonal course of $\bar{I}$, and $\text{d}\bar{I}/\text{d}t$ are affected.

\begin{figure}[ht!]
  \includegraphics[width=12cm,trim=0mm 0mm 0cm 0mm, clip]{figures/2022-03-22d_Ldvar_Tvar_ThatVar_PARext_IA_vs_DA/0D-Highlat_wconst_lext_Ldvar_Tvar_DIN015_CbasedIA_modular_24h_vs_0D-Highlat_wconst_lext_Ldvar_Tvar_DIN015_CbasedDA_modular_24h_cont_abio0_2y.png}
  \caption{Daily average irradiance and its temporal derivative, as extracted from the simulation outputs generated for T1. PAR:N (solid blue line) is the model version where the temporal derivative of irradiance is approximated numerically; PAR:A (dashed orange line): both irradiance and its temporal derivative are calculated analytically.\label{f.T2env}}
\end{figure}
    

% \begin{figure}[ht!]
% \centering
%   \setlength{\unitlength}{1mm}
%   \begin{picture}(120,70)(0,0)
%     \put(0,45){
%     \begin{overpic}[width=14cm,trim=0cm 16.8cm 0.0cm 5mm, clip]{figures/2022-02-10b_Ld1_T20_ThatVar_PARext_IA_vs_DA/0D-Highlat_wconst_lext_Ld1_T20_CbasedIA_modular_24h_vs_0D-Highlat_wconst_lext_Ld1_T20_CbasedDA_modular_24h_cont_abio1_1y.png}
%     \end{overpic}}
%     \put(0,22){
%     \begin{overpic}[width=14cm,trim=0cm 16.8cm 0.0cm 14mm, clip]{figures/2022-02-10c_Ldvar_T20_ThatVar_PARext_IA_vs_DA/0D-Highlat_wconst_lext_Ldvar_T20_CbasedIA_modular_24h_vs_0D-Highlat_wconst_lext_Ldvar_T20_CbasedDA_modular_24h_cont_abio1_1y.png}
%     \end{overpic}}
%     \put(0,0){
%     \begin{overpic}[width=14cm,trim=0cm 16.8cm 0.0cm 14mm, clip]{figures/2022-03-22d_Ldvar_Tvar_ThatVar_PARext_IA_vs_DA/0D-Highlat_wconst_lext_Ldvar_Tvar_CbasedIA_modular_24h_vs_0D-Highlat_wconst_lext_Ldvar_Tvar_CbasedDA_modular_24h_cont_abio1_1y.png}
%     \end{overpic}}
%     %\put(9,26){\scriptsize{\textsf{(a)}}}
%     %\put(69,26){\scriptsize{\textsf{(b)}}}
%   \put(0,0){\tiny \grid(140,70)(5,5)[0,0]}
%   \end{picture}
% 
% \caption{Results of T2: Carbon (left) and nitrogen (right) pools according to the IA (solid blue line) and DA (dashed orange line) approaches, (a-b) under constant $L_{\text{D}}=1$ and $T=20^{\circ}C$;  (c-d) variable $L_{\text{D}}$; (e-f) constant $T=20^{\circ}C$;  variable $L_{\text{D}}$ and$T$; . \onur{Again, I think no other panels required here}.\label{f.T2ares}}
% \end{figure}

%\subsection{Comparison of IA approach to the fully explicit DA approach (T2b)}

The IA and the fully explicit DA variants produce almost identical outputs (Fig.~\ref{f.T2res}-\ref{f.T2resbox}). This similarity is expected and in agreement with \citet{Ward2017}. On a closer look, some differences can be detected, such as a slightly higher $\text{Phy}_{\text{N}}$ at the peak of the spring bloom and slightly higher DON and $\text{Det}_{\text{N}}$ after the spring bloom simulated by the DA variant.  The differences are due to the re-allocation of part of the fluxes into and out of $\text{Phy}_{\text{N}}$ to DIN fluxes in Eq.~\eqref{eq:sdin3}.  They remain relatively small because (1) the time scale of the optimal regulation of N uptake in the DA variant is short relative to those of phytoplankton growth and the DIN-changes in our simulations, and (2) the strong interaction between phytoplankton and DIN leads to a negative feed-back between the deviations between the IA and DA variants and the extra DIN fluxes caused by variations in $Q_{\text{N}}$ in the IA variant.

\begin{figure}[htb!]
\includegraphics[width=14cm,trim=0cm 11mm 0.0cm 5mm, clip]{figures/2022-03-22d_Ldvar_Tvar_ThatVar_PARext_IA_vs_DA/0D-Highlat_wconst_lext_Ldvar_Tvar_DIN015_CbasedIA_modular_24h_vs_0D-Highlat_wconst_lext_Ldvar_Tvar_DIN015_CbasedDA_modular_24h_cont_abio1_2y.png}
\caption{Results of T2: Carbon (left) and nitrogen (right) pools according to the IA (solid blue line) and DA (dashed orange line) approaches under variable $L_{\text{D}}$ and $T$. \onur{We had planned to show box-whisker plots instead of this, but looking at that plot (Fig.~\ref{f.T2resbox}), I feel like we may want to keep this}.\label{f.T2res}}
\end{figure}

\begin{figure}[htb!]
\includegraphics[width=14cm,trim=0cm 0mm 0.0cm 0mm, clip]{figures/2022-03-22d_Ldvar_Tvar_ThatVar_PARext_IA_vs_DA/0D-Highlat_wconst_lext_Ldvar_Tvar_DIN015_Cbased_modular_24h_IAvsDA_boxplot_CN.png}
\caption{Results of T2: Carbon (left) and nitrogen (right) pools according to the IA (solid blue line) and DA (dashed orange line) approaches under variable $L_{\text{D}}$ and $T$. \onur{I'm not sure whether this is really better than Fig.~\ref{f.T2res}. Maybe we keep them both?}.\label{f.T2resbox}}
\end{figure}


\FloatBarrier
\subsection{T3: Comparing DA and IA variants in simulating in simulating multiple PFT's}\label{s.resT3}

For the case of multiple phytoplankton, Eq.~\ref{eq:sdin3} can be generalized as follows:
\begin{flalign}\label{eq:sdin4}
\frac{\text{d}\,\text{DIN}}{\text{d}\,t} &= \frac{\displaystyle F_{\text{DON}-\text{DIN}} - 
  \sum_j \left[ V_{\text{N}}^j - \text{Phy}_{\text{C}}^j \cdot
  \left( \frac{\partial Q_{\text{N}}^j}{\partial \bar{I}} \frac{\text{d}\, \bar{I}}{\text{d}\, t}
    + \frac{\partial Q_{\text{N}}^j}{\partial L_{\text{D}}} \frac{\text{d}\, L_{\text{D}}}{\text{d}\, t}
    + \frac{\partial Q_{\text{N}}^j}{\partial T} \frac{\text{d}\, T}{\text{d}\, t} \right) \right]} %
{\displaystyle 1+ \sum_j \left[ \text{Phy}_{\text{C}}^j\frac{\partial Q_{\text{N}}^j}{\partial \text{DIN}} \right]}
\end{flalign}
Where, $j$ indexes the different PFTs simulated \onur{maybe it's better to move it right after Eq.~\ref{eq:sdin3}}.

In Fig.~\ref{f.T3res} we present the results of a numerical experiment with 10 phytoplankton size classes, as simulated by IA and DA variants. Simulated C biomass of phytoplankton by the two variants are near-identical. Under the specific setup, we used for this experiment annual average concentrations decrease with cell size, and the larger classes exhibit a stronger seasonal variation. Under different environmental conditions, with respect to, e.g., the initial nutrient concentrations, and temporal stability, different outcomes can emerge \citep[see, e.g.,][]{Taherzadeh2017}, but such an exploration is outside of the scope of the current study.

\begin{figure}[ht!]
\includegraphics[width=14cm,trim=0cm 0mm 0.0cm 0mm, clip]{figures/2022-03-22f_Ldvar_Tvar_ThatVar_2P_PARext_IA_vs_DA/0D-Highlat_wconst_lext_Ldvar_Tvar_DIN015_10P_Cbased_modular_24h_IAvsDA_boxplot_size.png}
\caption{Results of T4: Box-whisker plots of C biomass of 10 phytoplankton size classes, as simulated by the IA (dark blue boxes) and DA (orange boxes) approaches under variable $L_{\text{D}}$ and $T$.\label{f.T3res}}
\end{figure}

%In Fig.~\ref{f.T3res} we present the results of a numerical experiment with 2 phytoplankton functional types, as simulated by IA and DA.The first type characterizes a fast growing `opportunist' species, and the second type characterizes a `gleaner' species, with lower growth rate but higher nutrient affinity. As expected, the first type reaches higher C-biomass during the spring bloom, whereas the second type maintains higher concentrations during summer, under low nutrient concentrations. The predictions by the two variants are almost identical.
% \begin{figure}[ht!]
% \includegraphics[width=14cm,trim=0cm 11mm 0.0cm 5mm, clip]{figures/2022-03-22e_Ldvar_Tvar_ThatVar_2P_PARext_IA_vs_DA/0D-Highlat_wconst_lext_Ldvar_Tvar_DIN015_2P_CbasedIA_modular_24h_vs_0D-Highlat_wconst_lext_Ldvar_Tvar_DIN015_2P_CbasedDA_modular_24h_cont_abio1_2y.png}
% \caption{Results of T3: Carbon (left) and nitrogen (right) pools according to the IA (solid blue line) and DA (dashed orange line) approaches under variable $L_{\text{D}}$ and $T$ and 2 phytoplankton types. \onur{Figure is for drafting purposes}.\label{f.T3res}}
% \end{figure}

As a technical note regarding the technical implementation is that, using the separate \verb|phy_Cbased.F90| and \verb|abio_Cbased.F90| modules, the number of phytoplankton types can be changed without needing to change and recompile the code, and by adjusting the configuration file in run time (see the \verb|fabm.yaml| examples in the \verb|testcases| folder that were employed to produce the results presented in this study), which is a feature enabled by the modularity of FABM.


%\include{res1D}

\FloatBarrier
% DISCUSSION
\section{Discussion}

Here a general Paragraph

\subsection{Re-establishing the Mass Balance \onur{alternative header: `An analytical modelling platform to study phytoplankton'. Any other suggestions?}}
As explained in section~\ref{S:DescFlux}, reducing the errors in mass balance for N required explicitly calculating the changes in N quota in time, i.e., $\textrm{d}Q/\textrm{d}t$, which in turn required calculation of individual components of this change driven by different environmental factors, namely, DIN, $\bar{I}$, T and $\textrm{L}_{\textrm{D}}$.  Under the setup considered for T2, changes occurring in DIN is clearly the dominant source of variation in Q, however contributions by other factors are non-negligible (Fig.~\ref{f.T2dQdt}. In other setups, relative importance of various environmental factors may be different. Relevance of these secondary factors to the elemental stoichiometry of phytoplankton is an often neglected aspect. Our mathematically explicit treatment of this issue in this study is expected to inspire and contribute to future endeavors to establish an analytical framework for investigating the mechanistic underpinnings of plankton physiology.

\begin{figure}[ht!]
\includegraphics[width=14cm,trim=0cm 0mm 0.0cm 0mm, clip]{figures/2022-03-22d_Ldvar_Tvar_ThatVar_PARext_IA_vs_DA/0D-Highlat_wconst_lext_Ldvar_Tvar_DIN015_CbasedIA_modular_24h_vs_0D-Highlat_wconst_lext_Ldvar_Tvar_DIN015_CbasedDA_modular_24h_cont_phy-4_2y.png}
\caption{For T2 experiment, total $\textrm{d}Q/\textrm{d}t$ and components of it through the changes in DIN, $\bar{I}$, $\textrm{L}_{\textrm{D}}$ and T.\label{f.T2dQdt}}
\end{figure}

\subsection{Computational Efficiency and Application Potential}
Our results demonstrate that a state variable that tracks the elemental content of plankton can be effectively removed, without leading to major issues in mass balance, in a 0D setup. Removing a state variable in such a setup does not seem to result in any advantages in computational efficiency, in comparison to a fully explicit dynamic variant, independent of the number of phytoplankton groups being simulated (Fig.~\ref{f.speed}). This is presumably because any potential reduction in computational costs owed to one less state variable in the instantaneous variant is  compensated by the  the additional logic and calculations required (see Appendix-\ref{S:Sol}).

\begin{figure}[ht!]
\includegraphics[width=14cm,trim=0cm 0mm 0.0cm 0mm, clip]{figures/time_of_run.png}
\caption{Scaling of simulation time with number of phytoplankton instances being simulated.\onur{I'm not sure whether we need to show this result, or it may be sufficient to mention the results, that there are no obvious speed differences between the IA and DA variants}.\label{f.speed}}
\end{figure}

The modelling framework, FABM, in which the model was implemented, allows seamless coupling of models with various hydrodynamical hosts \citep{Bruggeman2014}. We already attempted an application in a 1D setup using GOTM \citep{Burchard2006} as the hydrodynamical host, and found out that N is not conserved. This is understandable, because the spatial transport in a spatially explicit setup leads to variations in $Q$, which need to be compensated by variations in another state variable, like DIN, as we have done here for the case of variations in $Q$ that occurred in time (Eq.~\ref{eq:sdin2}). For instance, in a vertical setup with $z$ indicating the only spatial dimension, all environmental factors contributing to $\mathrm{d}Q/\mathrm{d}z$ need to be calculated, and included in the spatial flux terms of DIN, which may have to be handled within the hydrodynamical host (as would be the case in FABM application), depending on the modelling framework. Such additional fluxes need to be calculated for each spatial dimension across which material transport is described. It is not clear, whether the resulting model would be faster than the fully explicit variant: on one hand, reduced number of state variables in a spatially explicit setup means smaller transport matrices, on the other, calculation of fluxes associated with changes in $Q$ implies additional logic and calculations. As mentioned above, in the 0D setup, no clear performance advantage of the instantaneous acclimation approach was found presumably for this reason, however, in a spatially explicit setups, calculation of transport can easily become computationally more demanding than calculating the right hand sides of a biogeochemical model, therefore reduction of a state variable might indeed lead to some computational advantages.


%CONCLUSION
%\section{Conclusions}


%\codeavailability{For running the model and reproducing the results presented in this study, FABM and GOTM need to be downloaded and installed. See https://github.com/fabm-model/fabm/wiki/GOTM for the instructions. The FABM-NflexPD is available from the ‘Cbased’ branch of the git repository:\url{https://github.com/OnurKerimoglu/fabm-nflexpd.git}. Instructions for compiling FABM-NflexPD for GOTM-FABM and a 0D setup are provided in README.md. The `src’ folder contains the Fortran codes. The model was implemented as two separate modules: the `phy.F90' module that describes phytoplankton growth and the `abio.F90' module that describes everything other than phytoplankton. The phytoplankton module can reproduce the behavior of all three different variants considered in the manuscript through optional parameters. The `testcases’ folder contains the configuration (yaml) file that was used to produce the results presented in this manuscript, thereby providing examples of how each variant can be initiated.} %% use this section when having only software code available


%\dataavailability{TEXT} %% use this section when having only data sets available


%\codedataavailability{TEXT} %% use this section when having data sets and software code available


%\sampleavailability{TEXT} %% use this section when having geoscientific samples available


%\videosupplement{TEXT} %% use this section when having video supplements available

\clearpage
\appendix
\section{Analytical Solutions}\label{S:Sol}
To facilitate the solutions of the $\partial Q/\partial E$ ($E=\{\mathrm{DIN},\bar{I},\mathrm{T},\mathrm{L}_\mathrm{D}\}$ in Eq.~\ref{eq:sdin3}, we introduce a new variable $Z$ and re-write Eq.~\eqref{eq:Qopt} in terms of $Z$, as in \citet{Smith2016}:
\begin{flalign}
  \label{eq.Z}
  Q_{\text{N}} &= \frac{Q_{\text{N},0}}{2} \left( 1 + \sqrt{1 + \frac{1}{Z}}  \right), \qquad Z = \frac{Q_{\text{N},0}}{2}\left( \frac{\hat{\mu}_{\text{net}}}{\hat{V}_{\text{N}}} + \zeta_{\text{N}} \right) \\
  \label{eq.delQdelN}
 \frac{\partial Q_{\text{N}}}{\partial \text{DIN}} &= \frac{\partial Q_{\text{N}}}{\partial Z} \frac{\partial Z}{\partial \text{DIN}}, \qquad
 \frac{\partial Q_{\text{N}}}{\partial \bar{I}}  = \frac{\partial Q_{\text{N}}}{\partial Z} \frac{\partial Z}{\partial \bar{I}}, \qquad
 \frac{\partial Q_{\text{N}}}{\partial L_{\text{D}}} = \frac{\partial Q_{\text{N}}}{\partial Z} \frac{\partial Z}{\partial L_{\text{D}}}, \qquad
 \frac{\partial Q_{\text{N}}}{\partial T} = \frac{\partial Q_{\text{N}}}{\partial Z} \frac{\partial Z}{\partial T}
\end{flalign}
In Eqs.~\eqref{eq.delQdelN}, the common term $\partial Q_{\text{N}} / \partial Z$, as in S16, is:
\begin{equation} \label{eq:delQdelZ}
 \frac{\partial Q_{\text{N}}}{\partial Z} = \frac{-Q_{\text{N,0}}}{4 \cdot Z \cdot \sqrt{Z\cdot(1+Z)}}
\end{equation}
Recalling $\hat{V}_{\text{N}}$ from K21, Eq.~(17):
\begin{flalign}
  \hat{V}_{\text{N}} &= \frac{(1-f_{\text{A}})\hat{V}_{0} f_{\text{A}} \hat{A}_{0} \text{DIN}}{(1-f_{\text{A}})\hat{V}_{0} + f_{\text{A}} \hat{A}_{0} \text{DIN}}
  = \frac{\hat{V}_{0}\cdot \hat{A}_{0}\cdot\text{DIN}}{{(\sqrt{\hat{V}_{0}} + \sqrt{\hat{A}_{0}\cdot \text{DIN}})}^{2}}, \qquad
  f_{\text{A}} = \frac{1}{\displaystyle 1 + \sqrt{\frac{\hat{A}_{0}\cdot\text{DIN}}{\hat{V}_{0}}}} \\
  \intertext{We set the potential maximum rates of N and C acquisition numerically equal to the maximum-rate parameter $\mu_{0}$ \citep{Pahlowetal13}:}
  \label{eq:v0mu0}
  \hat{V}_{0} &= \hat{\mu}_{0} = \mu_{0} \cdot f(T) \qquad f(T) = \exp\left[ -\frac{E_{\text{a}}}{R}\left( \frac{1}{T / \text{K}} - \frac{1}{T_{\text{ref}} / \text{K}} \right) \right]
\end{flalign}
the partial derivative of $Z$ with respect to DIN is:
\begin{flalign}
  \frac{\partial Z}{\partial \text{DIN}} = \frac{\partial Z}{\partial \hat{V}_{\text{N}}} \frac{\mathrm{d}\, \hat{V}_{\text{N}}}{\mathrm{d}\, \text{DIN}}
  = -\frac{\hat{\mu}_{\text{net}} \cdot Q_{\text{N},0}}{2 \cdot\hat{A}_{0}\cdot \text{DIN}^{2}} \left( 1 + \sqrt{\frac{\hat{A}_{0}\cdot \text{DIN}}{\hat{V}_{0}}} \right)
\end{flalign}

For calculating the partial derivative of $Z$ with respect to $\bar{I}$, ${\partial Z}/{\partial \bar{I}}$, we recall $\hat{\mu}_{\text{net}}$, $L_{\text{I}}$ and $\hat{\theta}$ from K21, Eqs.~(20)--(23) \& (26):
\begin{flalign}
  \hat{\mu}_{\text{net}} &= L_{\text{D}}\hat{\mu}_{0} L_{\text{I}} (1-\zeta_{\text{Chl}}\hat{\theta})- R^{\text{Chl}}, \qquad
    R^{\text{Chl}} = f(T) \cdot R_{\text{M}}^{\text{Chl}} \zeta_{\text{Chl}}\hat{\theta} \\
  L_{\text{I}} &= 1 - \exp \left( \frac{-\alpha \hat{\theta} \bar{I}}{\hat{\mu}_{0}} \right), \qquad
  \hat{\theta} = \frac{1}{\zeta_{\text{Chl}}} + \frac{\hat{\mu}_{0}}{\alpha\cdot \bar{I}} \cdot (1 - W), \qquad
  W = \mathrm{W}_{0} \left[ \left( 1 + \frac{f(T) \cdot R_{\text{M}}^{{\text{Chl}}}}{L_{\text{D}} \cdot \hat{\mu}_{0}} \right)
   \cdot \exp \left( 1 + \frac{\alpha \cdot \bar{I}}{\hat{\mu}_{0} \cdot \zeta_{\text{Chl}}} \right) \right]
\end{flalign}
where $\mathrm{W}_{0}$ is the 0-branch of Lambert's W-function, and $\alpha$ and $\zeta_{\text{Chl}}$ are model parameters (initial Chl-specific slope of P-I curve and cost of Chl synthesis, respectively, Table~3 in K21). %  $\hat{\theta}$ is the optimal Chl density in the chloroplast.\\
$\partial Z/ \partial \bar{I}$ can then be derived by canonical application of the chain rule:
\begin{flalign}
 \frac{\partial Z}{\partial \bar{I}} &=
 \frac{\partial Z}{\partial \hat{\mu}_{\text{net}}} \left( \frac{\partial \hat{\mu}_{\text{net}}}{\partial\bar{I}}
   + \frac{\partial \hat{\mu}_{\text{net}}}{\partial\hat{\theta}} \frac{\mathrm{d}\, \hat{\theta}}{\mathrm{d}\,\bar{I}} \right)
 = \frac{\partial Z}{\partial \hat{\mu}_{\text{net}}} \frac{\partial \hat{\mu}_{\text{net}}}{\partial\bar{I}}
 \qquad  (\text{because}~ \frac{\partial{\hat{\mu}_{\text{net}}}}{\partial \hat{\theta}} = 0 ~ \text{by definition}) \\
 \frac{\partial Z}{\partial \hat{\mu}_{\text{net}}} &= \frac{Q_{\text{N,0}}}{2 \cdot \hat{V}_{\text{N}}} \\
 \frac{\partial \hat{\mu}_{\text{net}}}{\partial\bar{I}} &= L_{\text{D}} \cdot (1-\hat{\theta} \cdot \zeta_{\text{Chl}})  \cdot \alpha \cdot \hat{\theta} \cdot (1-L_{\text{I}}) \\
% \frac{\partial \hat{\mu}_{\text{net}}}{\partial\hat{\theta}} &= L_{\text{D}} \cdot [ \alpha \cdot \bar{I} \cdot ( 1 - L_{\text{I}} ) \cdot (1 - \zeta_{\text{Chl}} \cdot \hat{\theta}) - L_{\text{I}} \cdot \zeta_{\text{Chl}} \cdot \hat{\mu}_{0} ] - R_{\text{M}}^{\text{Chl}} \cdot \zeta_{\text{Chl}} \\
% \frac{\mathrm{d}\, \hat{\theta}}{\mathrm{d}\, \bar{I}} &= -\frac{\hat{\mu}_{0}}{\alpha\cdot \bar{I}^{2}} (1 - W) - \frac{W}{1 + W} \frac{1}{\bar{I}\cdot \zeta_{\text{Chl}}}
 \intertext{The day-length derivatives are}
 \frac{\partial Z}{\partial L_{\text{D}}} &= \frac{\partial Z}{\partial \hat{\mu}_{\text{net}}} \left( \frac{\partial \hat{\mu}_{\text{net}}}{\partial L_{\text{D}}}
   + \frac{\partial \hat{\mu}_{\text{net}}}{\partial\hat{\theta}} \frac{\mathrm{d}\, \hat{\theta}}{\mathrm{d}\,\bar{I}} \right)
 = \frac{\partial Z}{\partial \hat{\mu}_{\text{net}}} \frac{\partial \hat{\mu}_{\text{net}}}{\partial L_{\text{D}}} \\
 \frac{\partial \hat{\mu}_{\text{net}}}{\partial L_{\text{D}}} &= \hat{\mu}_{0} \cdot L_{\text{I}} \cdot (1 - \zeta_{\text{Chl}} \hat{\theta})
\end{flalign}
%The partial derivatives of $T$ Eq.~\eqref{eq:sdin3} are approximated numerically, as the analytical solution of $\partial Q_{\text{N}}/\partial T$ is unwieldy, given the temperature dependence of $\hat{\mu}_0$,  $\hat{V}_0$ and $R_{\text{M}}^{\text{Chl}}$, and numerous occurrence of these terms in intermediate quantities $\hat{\theta}$ and $L_{\text{I}}$ when $Q_{\text{N}}$ (Eq.~\ref{eq:Qopt}) is fully expanded. The numerical difference approximation is achieved by storing the value of $T$ in the previous integration step ($t=i-1$), calculating $Q_{\text{N}}$ based on this previous value of $T$ (but the values of DIN, $\bar{I}$ and $L_{\text{D}}$ in the current time step ($t=i$)), and substracting this from the current value of $Q_{\text{N}}$, i.e.,
%\begin{align}
% \frac{\partial Q_{\text{N}}}{\partial T} & \approx
% \frac{Q_{\text{N}}(\text{DIN}_{i},\ \bar{I}_{i},\ L_{\text{D}_i},\ T_{i}) - Q_{\text{N}}(\text{DIN}_{i},\ \bar{I}_{i},\ L_{\text{D}_i},T_{i-1})} {T_{i} - T_{i-1}} 
%\end{align}
The temperature-derivative of $Z$ is obtained via the derivatives with respect to $\hat{\mu}_{0}$, $\hat{V}_{0}$ and $R^{\text{Chl}}$:
\begin{flalign}
  \begin{split}
    \frac{\partial Z}{\partial T}
    &=   \frac{\partial Z}{\partial \hat{\mu}_{\text{net}}}
      \left( \frac{\partial \hat{\mu}_{\text{net}}}{\partial \hat{\mu}_{0}} \frac{\partial \hat{\mu}_{0}}{\partial T}
        + \frac{\partial \hat{\mu}_{\text{net}}}{\partial R^{\text{Chl}}} \frac{\partial R^{\text{Chl}}}{\partial T} \right)
      + \frac{\partial Z}{\partial \hat{V}_{\text{N}}} \frac{\partial \hat{V}_{\text{N}}}{\partial \hat{V}_{0}} \frac{\partial \hat{V}_{0}}{\partial T} \\ &
    = \left[ \frac{\partial Z}{\partial \hat{\mu}_{\text{net}}}
      \left( \frac{\partial \hat{\mu}_{\text{net}}}{\partial \hat{\mu}_{0}} \cdot \hat{\mu}_{0} - R^{\text{Chl}} \right)
      + \frac{\partial Z}{\partial \hat{V}_{0}} \cdot \hat{V}_{0}
    \right] \frac{1}{f(T)} \frac{\text{d}\,f(T)}{\text{d}\,T}
  \end{split}
  \\
  \frac{\partial \hat{\mu}_{net}}{\partial \hat{\mu}_{0}} &= L_{\text{D}} \cdot (1 - \zeta_{\text{Chl}} \hat{\theta})
  \left[ L_{\text{I}} - (1 - L_{\text{I}}) \frac{\alpha \cdot \bar{I}}{\hat{\mu}_{0}} \hat{\theta} \right] \qquad
  \frac{\partial Z}{\partial \hat{V}_{0}} = -\hat{\mu}_{\text{net}} \frac{Q_{\text{N},0}}{2 \hat{\mu}_{0} \sqrt{\hat{\mu}_{0} \cdot \hat{V}_{\text{N}}}} \\
  \frac{1}{f(T)} \frac{\text{d}\,f(T)}{\text{d}\,T} &= \frac{E_{\text{a}}}{R\cdot {(T / \text{K})}^{2}}
\end{flalign}

We would like to clarify that
%(1) the DIN in the denominator in the solution provided by S16 (their Eq.~A-5) was a typo; 
(1) replacement of $\hat{\mu}_g$ (in S16, $\hat{\mu}^I$) with $\hat{\mu}_{\text{net}}$ in our model (see K21) results in the appearance of $(1-\hat{\theta} \cdot \zeta_{\text{Chl}})$ when computing $\partial \hat{\mu}_{\text{net}} / \partial \bar{I}$; 
(2) $L_{\text{D}}$ used to be implicit in S16, now it's explicit (K21 Eq.~21) therefore it appears for $\partial\hat{\mu}_{\text{net}} / \partial \bar{I}$ unlike in S16.

% \section{Relevance of changes in quota for the loss processes}\label{S:appQ4loss}  %% Appendix-A
% \onur{Not sure whether we want to keep this}
% %In our model, we considered a single loss process for phytoplankton.
% %Recalling from K21 (Table 1) that the associated N flux term was expressed as
% %\begin{flalign}
% % F_{\text{Phy}_{\text{N}}-\text{Det}_{\text{N}}} &= m \cdot \text{Phy}_{\text{N}}^2 &&
% %\end{flalign}
% %and considering that $\text{Phy}_{\text{N}}=\text{Phy}_{\text{C}} \cdot Q_{\text{N}}$, it can be thought that the changes in $Q_{\text{N}}$ needs to be taken into account for calculating the flux between phytoplankton and detrital N due to mortality.
% Potential links between cellular quotas and loss processes can be assessed by analyzing the components of the changes in $\text{Phy}_{\text{N}}$ occurring due to loss terms, as was previously done for the case of growth (Eq.~\eqref{eq:dphyNdt}). Observing that the only loss process we considered here is mortality:
% \begin{flalign}
%   \frac{\text{d}\,\text{Phy}_{\text{N}}}{\text{d}\,t} \bigg\rvert_L
%   &=\frac{\text{d}\,\text{Phy}_{\text{C}} Q_{\text{N}}}{\text{d}\,t} \bigg\rvert_M
%   = Q_{\text{N}} \frac{\text{d}\,\text{Phy}_{\text{C}}}{\text{d}\,t} \bigg\rvert_M + \text{Phy}_{\text{N}}\frac{\text{d}\,\text{Q}_{\text{N}}}{\text{d}\,t} \bigg\rvert_M
% \end{flalign}
% Here, in the first term, $\frac{\text{d}\,\text{Phy}_{\text{C}}}{\text{d}\,t} \rvert_M$ represents the C-based mortality rate ($=m \cdot \text{Phy}_{\text{C}}^2$, Eq.~\eqref{eq:mortC}). In the second term, $\frac{\text{d}\,\text{Q}_{\text{N}}}{\text{d}\,t} \rvert_M$ represents the changes occurring in $Q_{\text{N}}$, however, due to mortality only. As in our model, we did not consider any effect of mortality on quota (assuming that when a cell dies, it dies altogether and all of its contents become detritus), this term collapses to 0, therefore reducing the associated flux term to:
% \begin{flalign}
% F_{\text{Phy}_{\text{N}}-\text{Det}_{\text{N}}} &=
% Q_{\text{N}} \frac{\text{d}\,\text{Phy}_{\text{C}}}{\text{d}\,t} \bigg\rvert_M = Q_{\text{N}} \cdot m \cdot \text{Phy}_{\text{C}}^2
% \end{flalign}
% However, in any other model that may consider processes that may affect the cellular quotas, such as excretion of C-rich, N-poor substances, associated consequences on various elemental fluxes can be taken into account following this framework.

% \section{On the differences between the IA and DA variants}\label{S:appIADA}  %% Appendix-B
% \onur{Markus, would you like to draft this section?}

\noappendix%% use this to mark the end of the appendix section. Otherwise the figures might be numbered incorrectly (e.g. 10 instead of 1).

%% Regarding figures and tables in appendices, the following two options are possible depending on your general handling of figures and tables in the manuscript environment:

%% Option 1: If you sorted all figures and tables into the sections of the text, please also sort the appendix figures and appendix tables into the respective appendix sections.
%% They will be correctly named automatically.

%% Option 2: If you put all figures after the reference list, please insert appendix tables and figures after the normal tables and figures.
%% To rename them correctly to A1, A2, etc., please add the following commands in front of them:

%\appendixfigures%% needs to be added in front of appendix figures

%\appendixtables%% needs to be added in front of appendix tables

%% Please add \clearpage between each table and/or figure. Further guidelines on figures and tables can be found below.

%\authorcontribution{} %% this section is mandatory

%\competinginterests{No competing interests are present.} %% this section is mandatory even if you declare that no competing interests are present

%\disclaimer{TEXT} %% optional section

%\begin{acknowledgements}
%OK, SLS and PA were supported through a bilateral research project, funded jointly by the German Research Foundation DFG (grant no. KE 1970/2-1, P.I.: OK) and the Japan Society for the Promotion of Science, JSPS (P.I.: SLS).We acknowledge the developers of the open-source software used in this study, foremost FABM and GOTM.
%\end{acknowledgements}



%% REFERENCES

%% The reference list is compiled as follows:

% \begin{thebibliography}{}
%
% \bibitem[AUTHOR(YEAR)]{LABEL1}
% REFERENCE 1
%
% \bibitem[AUTHOR(YEAR)]{LABEL2}
% REFERENCE 2
%
% \end{thebibliography}

%% Since the Copernicus LaTeX package includes the BibTeX style file copernicus.bst,
%% authors experienced with BibTeX only have to include the following two lines:
%%
\bibliographystyle{copernicus}
\bibliography{NflexPD_2_0.bib}
%%
%% URLs and DOIs can be entered in your BibTeX file as:
%%
%% URL = {http://www.xyz.org/~jones/idx_g.htm}
%% DOI = {10.5194/xyz}


%% LITERATURE CITATIONS
%%
%% command                        & example result
%% \citet{jones90}|               & Jones et al. (1990)
%% \citep{jones90}|               & (Jones et al., 1990)
%% \citep{jones90,jones93}|       & (Jones et al., 1990, 1993)
%% \citep[p.~32]{jones90}|        & (Jones et al., 1990, p.~32)
%% \citep[e.g.,][]{jones90}|      & (e.g., Jones et al., 1990)
%% \citep[e.g.,][p.~32]{jones90}| & (e.g., Jones et al., 1990, p.~32)
%% \citeauthor{jones90}|          & Jones et al.
%% \citeyear{jones90}|            & 1990



%% FIGURES

%% When figures and tables are placed at the end of the MS (article in one-column style), please add \clearpage
%% between bibliography and first table and/or figure as well as between each table and/or figure.

% The figure files should be labelled correctly with Arabic numerals (e.g. fig01.jpg, fig02.png).


%% ONE-COLUMN FIGURES

%%f
%\begin{figure}[t]
%\includegraphics[width=8.3cm]{FILE NAME}
%\caption{TEXT}
%\end{figure}
%
%%% TWO-COLUMN FIGURES
%
%%f
%\begin{figure*}[t]
%\includegraphics[width=12cm]{FILE NAME}
%\caption{TEXT}
%\end{figure*}
%
%
%%% TABLES
%%%
%%% The different columns must be seperated with a & command and should
%%% end with \\ to identify the column brake.
%
%%% ONE-COLUMN TABLE
%
%%t
%\begin{table}[t]
%\caption{TEXT}
%\begin{tabular}{column = lcr}
%\tophline
%
%\middlehline
%
%\bottomhline
%\end{tabular}
%\belowtable{} % Table Footnotes
%\end{table}
%
%%% TWO-COLUMN TABLE
%
%%t
%\begin{table*}[t]
%\caption{TEXT}
%\begin{tabular}{column = lcr}
%\tophline
%
%\middlehline
%
%\bottomhline
%\end{tabular}
%\belowtable{} % Table Footnotes
%\end{table*}
%
%%% LANDSCAPE TABLE
%
%%t
%\begin{sidewaystable*}[t]
%\caption{TEXT}
%\begin{tabular}{column = lcr}
%\tophline
%
%\middlehline
%
%\bottomhline
%\end{tabular}
%\belowtable{} % Table Footnotes
%\end{sidewaystable*}
%
%
%%% MATHEMATICAL EXPRESSIONS
%
%%% All papers typeset by Copernicus Publications follow the math typesetting regulations
%%% given by the IUPAC Green Book (IUPAC: Quantities, Units and Symbols in Physical Chemistry,
%%% 2nd Edn., Blackwell Science, available at: http://old.iupac.org/publications/books/gbook/green_book_2ed.pdf, 1993).
%%%
%%% Physical quantities/variables are typeset in italic font (t for time, T for Temperature)
%%% Indices which are not defined are typeset in italic font (x, y, z, a, b, c)
%%% Items/objects which are defined are typeset in roman font (Car A, Car B)
%%% Descriptions/specifications which are defined by itself are typeset in roman font (abs, rel, ref, tot, net, ice)
%%% Abbreviations from 2 letters are typeset in roman font (RH, LAI)
%%% Vectors are identified in bold italic font using \vec{x}
%%% Matrices are identified in bold roman font
%%% Multiplication signs are typeset using the LaTeX commands \times (for vector products, grids, and exponential notations) or \cdot
%%% The character * should not be applied as mutliplication sign
%
%
%%% EQUATIONS
%
%%% Single-row equation
%
%\begin{equation}
%
%\end{equation}
%
%%% Multiline equation
%
%\begin{align}
%& 3 + 5 = 8\\
%& 3 + 5 = 8\\
%& 3 + 5 = 8
%\end{align}
%
%
%%% MATRICES
%
%\begin{matrix}
%x & y & z\\
%x & y & z\\
%x & y & z\\
%\end{matrix}
%
%
%%% ALGORITHM
%
%\begin{algorithm}
%\caption{...}
%\label{a1}
%\begin{algorithmic}
%...
%\end{algorithmic}
%\end{algorithm}
%
%
%%% CHEMICAL FORMULAS AND REACTIONS
%
%%% For formulas embedded in the text, please use \chem{}
%
%%% The reaction environment creates labels including the letter R, i.e. (R1), (R2), etc.
%
%\begin{reaction}
%%% \rightarrow should be used for normal (one-way) chemical reactions
%%% \rightleftharpoons should be used for equilibria
%%% \leftrightarrow should be used for resonance structures
%\end{reaction}
%
%
%%% PHYSICAL UNITS
%%%
%%% Please use \unit{} and apply the exponential notation

\end{document}

% Local Variables:
% TeX-engine: default
% End:
