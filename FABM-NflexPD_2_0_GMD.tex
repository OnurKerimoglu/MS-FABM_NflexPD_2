%% Copernicus Publications Manuscript Preparation Template for LaTeX Submissions
%% ---------------------------------
%% This template should be used for copernicus.cls
%% The class file and some style files are bundled in the Copernicus Latex Package, which can be downloaded from the different journal webpages.
%% For further assistance please contact Copernicus Publications at: production@copernicus.org
%% https://publications.copernicus.org/for_authors/manuscript_preparation.html


%% Please use the following documentclass and journal abbreviations for preprints and final revised papers.

%% 2-column papers and preprints
\documentclass[gmd, manuscript]{copernicus}
%\documentclass[gmd, manuscript, draft]{copernicus}

%% Journal abbreviations (please use the same for preprints and final revised papers)

% Advances in Geosciences (adgeo)
% Advances in Radio Science (ars)
% Advances in Science and Research (asr)
% Advances in Statistical Climatology, Meteorology and Oceanography (ascmo)
% Annales Geophysicae (angeo)
% Archives Animal Breeding (aab)
% ASTRA Proceedings (ap)
% Atmospheric Chemistry and Physics (acp)
% Atmospheric Measurement Techniques (amt)
% Biogeosciences (bg)
% Climate of the Past (cp)
% DEUQUA Special Publications (deuquasp)
% Drinking Water Engineering and Science (dwes)
% Earth Surface Dynamics (esurf)
% Earth System Dynamics (esd)
% Earth System Science Data (essd)
% E&G Quaternary Science Journal (egqsj)
% European Journal of Mineralogy (ejm)
% Fossil Record (fr)
% Geochronology (gchron)
% Geographica Helvetica (gh)
% Geoscience Communication (gc)
% Geoscientific Instrumentation, Methods and Data Systems (gi)
% Geoscientific Model Development (gmd)
% History of Geo- and Space Sciences (hgss)
% Hydrology and Earth System Sciences (hess)
% Journal of Bone and Joint Infection (jbji)
% Journal of Micropalaeontology (jm)
% Journal of Sensors and Sensor Systems (jsss)
% Magnetic Resonance (mr)
% Mechanical Sciences (ms)
% Natural Hazards and Earth System Sciences (nhess)
% Nonlinear Processes in Geophysics (npg)
% Ocean Science (os)
% Primate Biology (pb)
% Proceedings of the International Association of Hydrological Sciences (piahs)
% Scientific Drilling (sd)
% SOIL (soil)
% Solid Earth (se)
% The Cryosphere (tc)
% Weather and Climate Dynamics (wcd)
% Web Ecology (we)
% Wind Energy Science (wes)


%% \usepackage commands included in the copernicus.cls:
%\usepackage[german, english]{babel}
%\usepackage{tabularx}
%\usepackage{cancel}
%\usepackage{multirow}
%\usepackage{supertabular}
%\usepackage{algorithmic}
%\usepackage{algorithm}
%\usepackage{amsthm}
%\usepackage{float}
%\usepackage{subfig}
%\usepackage{rotating}

%Additional packages: 
\usepackage{mathrsfs} %provides calligraphic Fonts
\usepackage{placeins}
\allowdisplaybreaks

\graphicspath{
%{./figures/common/}
%{./figures/v_10_13_solitary/}
{./figures/}
}

\newcommand{\onur}[1]{\textcolor{blue}{\{Onur: #1\}}}

\begin{document}
\title{FABM-NflexPD 2.0: An instantenous acclimation approach for modelling carbon, nitrogen [and phosphorus ?] cycles}

% \Author[affil]{given_name}{surname}

\Author[1]{Onur}{Kerimoglu}
\Author[2]{Prima}{Anugerahanti}
\Author[2]{S. Lan}{Smith}

\affil[1]{Institute for Chemistry and Biology of the Marine Environment, University of Oldenburg, Oldenburg, Germany}
%\affil[2]{Helmholtz Center for Coastal Research, Germany}
\affil[2]{Earth SURFACE Research Center, Research Institute for Global Change, JAMSTEC, Yokosuka, Japan}

%% The [] brackets identify the author with the corresponding affiliation. 1, 2, 3, etc. should be inserted.

%% If an author is deceased, please mark the respective author name(s) with a dagger, e.g. "\Author[2,$\dag$]{Anton}{Aman}", and add a further "\affil[$\dag$]{deceased, 1 July 2019}".

%% If authors contributed equally, please mark the respective author names with an asterisk, e.g. "\Author[2,*]{Anton}{Aman}" and "\Author[3,*]{Bradley}{Bman}" and add a further affiliation: "\affil[*]{These authors contributed equally to this work.}".


\correspondence{Onur Kerimoglu (kerimoglu.o@gmail.com)}

\runningtitle{FABM-NflexPD 2.0}

\runningauthor{Kerimoglu et al.}


\received{}
\pubdiscuss{} %% only important for two-stage journals
\revised{}
\accepted{}
\published{}

%% These dates will be inserted by Copernicus Publications during the typesetting process.


\firstpage{1}

\maketitle


\begin{abstract}

\end{abstract}


%\copyrightstatement{TEXT}

%INTRODUCTION
\introduction%% \introduction[modified heading if necessary]

%\subsection{Autotroph Acclimation and Biogeochemical Cycles} 

%\subsection{Instantaneous acclimation approach}
\subsection{Background and Objectives}
Recently, \citet{Kerimoglu2021} introduced FABM-NflexPD 1.0, which is the FABM \citep[Framework for Aquatic Biogeochemical Models][]{Bruggeman2014} implementation of an instantaneous acclimation model of phytoplankton growth originally developed by \citet{Smith2016}. That model version can only resolve N in the Dissolved Inorganic Material (DIM) pool, which may be sufficient for some ecological applications, %\citep[e.g.,][]{Anugerahanti2021},
but not for applications that require resolution of multiple nutirents and carbon in DIM form.

%\subsection{Objectives of this study}
Here we introduce FABM-NflexPD 2.0, which can resolve C, N and P in DIM pool. We present:
\begin{itemize}
 \item a detailed description of the current, C-based version with reference to the previous, N-based version (see below)
 \item extension of the model with DIC, to enable tracing the conservation of C
 \item description of the model extended with limitation by other elements, exemplified for P
 \item an assessment of the conservation of C and nutrients by the model
 \item \onur{not sure if really needed} comparison of the behavior of the previous and current C- and N- based model versions in a 0D setup  % Not sure also whether the two versions will produce quantitatively identical results because of the approximations and potential omission of a term in the N-based version (see below)}
 \item \onur{later, if the 0D works without problems} application of the C-N-P model in an idealisitc 1D setup and formal assessment of model behavior (e.g., mass conservation)
\end{itemize}

% \begin{figure}[htb]
%   \centering
%   \includegraphics[width=12cm]{?.pdf}
%   \caption{\label{?}}
% \end{figure}

%MODEL DESCRIPTION
\section{Model Description}
In this section, we will highlight the differences and similarities between the present, `C-based' version of the model from the earlier version of the `N-based' model version presented by \citet{Kerimoglu2021}, K21 in short hereafter. 

\subsection{Model setup and operation}\label{S:DescSetup}
The model is operated in a spatially homogeneous `box' setup, using the 0d driver of FABM \citep{Bruggeman2014}. As simplifications, we assume that temperature is fixed at 10$\degree$C, fractional day length, $L_D$ is unity, and ignore light attenuation and albedo effects. See the end of Section~\ref{S:DescFlux} for the description of irradiance. Numerical solutions are obtained using 4$^\text{th}$ order Runge-Kutta method with a time step of 60 seconds. 

\subsection{C and N content of Phytoplankton}
In this version of the model, C content of phytoplankton (Phy$_\text{C}$) is dynamically tracked (thus, C-based) instead of the N content (Phy$_\text{N}$) as was the case in the earlier (thus, N-based) version. Wherever required, Phy$_\text{N}$ is calculated as:
\begin{equation} \label{eq:Q}
Phy_\text{N} = Phy_\text{C} \cdot Q_\text{N}  
\end{equation}

$Q_\text{N}$ is assumed to instantaneously adjust to a balanced-growth optimum value, as determined by the nutrient uptake in the protopolast, $\hat{V}_\text{N}$ and net photosynthesis in the chloroplast, $\hat{\mu}_{net}$ (Eq.~10 in K21):
\begin{equation}\label{eq:Qopt}
 Q_\text{N}^o= \frac{Q_\text{N,0}}{2} \left[1+\sqrt{1+\frac{2}{Q_\text{N,0}{({\hat{\mu}_{\text{net}}}/{\hat{V}_\text{N}}+\zeta_\text{N} )}}} \right] \hfill \{\text{IA}\}
\end{equation}
where $Q_\text{N,0}$ and $\zeta_\text{N}$ are model parameters (subsistence N quota, and cost of N uptake, respectively, see Table~3 in K21). The source term for $Phy_\text{C}$ is given by:
\begin{equation} \label{eq:sPhyC}
s(Phy_{\text{C}}) = F_{DIC-Phy_\text{C}} - F_{Phy_{\text{C}}-Det_{\text{C}}}
\end{equation}
which is identical to the source term of the `DA' variant (Eq.~1b in K21). Here, $F_{DIC-Phy_\text{C}}$ and $F_{Phy_{\text{C}}-Det_{\text{C}}}$ are the fluxes from Dissolved Inorganic Carbon (DIC) to phytoplankton, and from phytoplankton to detrital C (Det$_\text{C}$), respectively. The first term in Eq.~\ref{eq:sPhyC} represents net phytoplankton growth, and as before (Eq.~8 in K21), it is given by the product of Phy$_\text{C}$ with net growth rate, $\mu$: 
\begin{equation} \label{eq:fdicphyc}
 F_{DIC-Phy_\text{C}} = \mu \cdot Phy_\text{C}
\end{equation}
This term is used in this updated model version also to track the DIC concentration, which in turn enables tracing the total C in the system:
\begin{equation} \label{eq:dic}
  s(\text{DIC}) = \underbrace{F_{DOC-DIC}}_\textrm{Remineralization} - \underbrace{F_{DIC-Phy_{\text{C}}}}_\textrm{Net C uptake}
\end{equation}
Note that representing the C uptake and respiration terms with a lumped term $F_{DIC-Phy_{\text{C}}}$ in Eq.~\ref{eq:dic} is based on the assumption that the respiration terms (see Eq.~7 in K21) are added back to DIC, and would have to be seperated if CO2 was explicitly resolved. Remineralization flux, $F_{DOC-DIC}$ is calculated as a order reaction as in K21, which was previously only to trace the DOC (K21).

The second term in Eq.~\ref{eq:sPhyC}, $F_{Phy_\text{C}-Det_\text{C}}$, representing the mortality of phytoplankton, is calculated as a quadratic rate like in K21 (Table 1), but now it is directly based on $\text{Phy}_\text{C}$ concentration, i.e.,  
\begin{flalign}\label{eq:mortC}
F_{Phy_\text{C}-Det_\text{C}} &= m_\text{C} \cdot \text{Phy}_\text{C}^2
\end{flalign}
with $m_\text{C}$ [\unit{m^3\ mmolC^{-1}\ d^{-1}}] being the C-based specific mortality rate. Then the N counterpart, $F_{Phy_\text{N}-Det_\text{N}}$ is found by multiplying this term by $Q_\text{N}$ (see also Appendix~\ref{S:Q4loss}).

%given by the division of N flux from phytoplankton to detritus, $F_{Phy_\text{N}-Det_\text{N}}$ by $Q_\text{N}$

\subsection{Flux of nutrients between DIM and Phytoplankton}\label{S:DescFlux}
As in K21, the source term for the nutrients in DIM pool are described as the remineralization fluxes from the nutrients in dissolved Organic form (e.g., DON), $F_{DON-DIN}$ (Table 1, K21), minus the fluxes between the DIM (e.g., DIN) and phytoplankton ($F_{DIN-Phy_\text{N}}$): 
\begin{equation} \label{eq:sdin}
  s(\text{DIN}) = \frac{\text{d}\text{DIN}}{\text{d}t} = F_{DON-DIN} - F_{DIN-Phy_\text{N}}
\end{equation}
$F_{DON-DIN}$ is calculated like the $F_{DOC-DIC}$ explained above. $F_{DIN-Phy_\text{N}}$, in other words, gross growth of phytoplankton N biomass, i.e., $\frac{\text{d}Phy_\text{N}}{\text{d}t} \big\rvert_G$, can be expressed as:
\begin{equation} \label{eq:dphyNdt}
  F_{DIN-Phy_\text{N}} = \frac{\text{d}Phy_\text{N}}{\text{d}t} \bigg\rvert_G=\frac{\text{d}(Phy_\text{C} \cdot Q)}{\text{d}t} \bigg\rvert_G = Q_\text{N} \frac{\text{d} Phy_\text{C}}{\text{d} t} \bigg\rvert_G + Phy_\text{C} \frac{\text{d} Q_\text{N}}{\text{d} t} 
\end{equation}
Here, $\frac{\text{d} Phy_\text{C}}{\text{d} t} \big\rvert_G$ corresponds to gross growth of phytoplankton C biomass, i.e., $F_{DIC-Phy_\text{C}}$ in Eq.~\ref{eq:sPhyC}. Substituting therefore Eq.~\ref{eq:fdicphyc} in Eq.~\ref{eq:dphyNdt}:
\begin{equation} \label{eq:dphyNdt2}
  \frac{\text{d}Phy_\text{N}}{\text{d}t} \bigg\rvert_G= Q_\text{N} \cdot \mu \cdot Phy_\text{C} + Phy_\text{C} \frac{\text{d} Q_\text{N}}{\text{d} t} 
\end{equation}
where, assuming balanced growth \citep{Burmaster1979}, i.e., $V_N = Q_N \cdot \mu$  (K21, Eq.~6) the first term on the right hand side of Eq.~\ref{eq:dphyNdt2} can be replaced with $V_N \cdot Phy_\text{C}$, i.e., 
\begin{equation} \label{eq:dphyNdt3}
  \frac{\text{d}Phy_\text{N}}{\text{d}t} \bigg\rvert_G=  Phy_\text{C} \left( V_\text{N} + \frac{\text{d} Q_\text{N}}{\text{d} t} \right)
\end{equation}

%\onur{In the N-based version, we had (K21, Eq.~5): $F_{DIN-%Phy_{\text{N}}} = V_N \cdot Phy_{\text{C}}$. i.e., excluding the $%\frac{\text{d} Q_\text{N}}{\text{d} t}$ term in Eq.~\ref{eq:dphyNdt3}. %As we discussed, this apparent difference %(apparent, because maybe there is a mistake in the notation? Also note that $F_{DIN-Phy_{\text{N}}}$ is not used anywhere, and in the C-based version of the model it is only saved as a diagnostic quantity) 
%doesn't violate the internal consistency of the N-based version, but it would be great if we could explain what this really is. For instance, is this a `real deviation' in model formulations that contributes (among other sources) to quantitative differences in results?}

According to Eq.~\ref{eq:dphyNdt3}, the N flux between the DIM pool and phytoplankton can be intuitively understood as the sum of uptake due to growth of C biomass, and the change in the N quota. % It should be noted that this cumulative change can become negative. (?)

Given that the terms $\hat{V}_\text{N}$ and $\hat{\mu}_{net}$ required for calculating $Q_\text{N}$ (Eq.~\ref{eq:Qopt}) are respectively functions of DIN (Eq.~16 in K21) and daytime average irradiance, $\bar{I}$ (Eqs.~20-22 in K21), the total time derivative of $Q_\text{N}$ in Eq.~\ref{eq:dphyNdt3} can be computed as the sum of partial derivatives with respect to DIN and $\bar{I}$ using chain rule: 
\begin{equation} \label{eq:dqdt}
 \frac{\text{d} Q_\text{N}}{\text{d} t} = \frac{\partial Q_\text{N}}{\partial \text{DIN}} \frac{\text{d} \text{DIN}}{\text{d} t} +  \frac{\partial Q_\text{N}}{\partial \bar{I}} \frac{\text{d} \bar{I}}{\text{d} t} 
\end{equation}
Substituting Eq.~\ref{eq:dqdt} in Eq.~\ref{eq:dphyNdt3}, and subsequently Eq.~\ref{eq:dphyNdt3} in Eq.~\ref{eq:sdin}:
\begin{equation}\label{eq:sdin2}
 \frac{\text{d}\text{DIN}}{\text{d}t} = F_{DON-DIN} - Phy_{\text{C}} \left(V_N + \frac{\partial Q_\text{N}}{\partial \text{DIN}} \frac{\text{d} \text{DIN}}{\text{d} t} +  \frac{\partial Q_\text{N}}{\partial \bar{I}} \frac{\text{d} \bar{I}}{\text{d} t} \right)
\end{equation}
and reorganizing:
%intermediate step:
% \begin{equation}\label{eq:sdin2.5}
%  \frac{\text{d}\text{DIN}}{\text{d}t} \left( 1+Phy_{\text{C}}\frac{\partial Q_\text{N}}{\partial \text{DIN}} \right) = F_{DON-DIN} - Phy_{\text{C}} \left(V_N +  \frac{\partial Q_\text{N}}{\partial \bar{I}} \frac{\text{d} \bar{I}}{\text{d} t} \right)\\
% \end{equation}
\begin{equation}\label{eq:sdin3}
 s(\text{DIN}) = \frac{\text{d}\text{DIN}}{\text{d}t} = \frac{F_{DON-DIN} - Phy_{\text{C}} \left(V_N +  \frac{\partial Q_\text{N}}{\partial \bar{I}} \frac{\text{d} \bar{I}}{\text{d} t} \right)}{ 1+Phy_{\text{C}}\frac{\partial Q_\text{N}}{\partial \text{DIN}}}
\end{equation}
As a technical remark regarding the FABM implementation of the model: Eq.~\ref{eq:sdin3} requires combination of terms, which, in K21, used to be calculated by separate abiotic ($F_{DON-DIN}$) and phytoplankton modules (all other terms), therefore in the current implementation, in order to avoid a `circular dependency' error, an intermediate module collects the necessary terms from the two modules, and sets the right hand sides for the DIN at once.

In Eq.~\ref{eq:sdin3}, partial derivatives of $Q_\text{N}$ with respect to DIN and $\bar{I}$ are obtained by canonical application of the chain rule. As in \citet{Smith2016}, defining $Z = (Q_\text{N,0}/2)\left(\hat{\mu}_\text{net}/\hat{V}_\text{N}+\zeta_\text{N} \right)$ in Eq.~\ref{eq:Qopt}:
\begin{flalign}\label{eq.delQdelN}
 \frac{\partial Q_\text{N}}{\partial \text{DIN}} &= \frac{\partial Q_\text{N}}{\partial Z} \frac{\partial Z}{\partial \text{DIN}}
\end{flalign}
\begin{flalign}\label{eq.delQdelI} 
 \frac{\partial Q_\text{N}}{\partial \bar{I}}  &= \frac{\partial Q_\text{N}}{\partial Z} \frac{\partial Z}{\partial \bar{I}}
\end{flalign}
In Eqs.~\ref{eq.delQdelN}-\ref{eq.delQdelI}, the common term $\frac{\partial Q_\text{N}}{\partial Z}$, as in S16, yields \onur{I verified this}:
\begin{equation} \label{eq:delQdelZ}
 \frac{\partial Q_\text{N}}{\partial Z} = \frac{-Q_\text{N,0}}{4 \cdot Z \cdot \sqrt{Z\cdot(1+Z)}}
\end{equation}
Recalling $\hat{V}_\text{N}$ from K21, Eq.~17:
\begin{flalign}
 \hat{V}_\text{N} &= \frac{(1-f_A)\hat{V}_0 f_A \hat{A}_0 \text{DIN}}{(1-f_A)\hat{V}_0 + f_A \hat{A}_0 \text{DIN}}
\end{flalign}
the partial derivative of $Z$ with respect to DIN can found to be similar to that in S16, with the only exception that their $\hat{\mu}^I$ is replaced by our $\hat{\mu}_{net}$:
\begin{flalign}
 \frac{\partial Z}{\partial \text{DIN}} =
 \frac{-\hat{\mu}_{net} \cdot Q_\text{N,0}/2}{\hat{V}_\text{N} \cdot \text{DIN}} \left( 1 - \frac{\hat{V}_\text{N}}{\hat{V}_\text{N,0}} - \frac{\hat{V}_\text{N}}{\sqrt{\hat{V}_\text{N,0} \cdot \hat{A}_\text{N,0} \cdot \text{DIN}}} \right)
\end{flalign}
\onur{I could not reproduce the solution above. The one I found is:
\begin{flalign}
 \frac{\partial Z}{\partial \text{DIN}} =
 \frac{\partial Z}{\partial \hat{V}_\text{N}} \frac{\partial \hat{V}_\text{N}}{\partial \text{DIN}} &=
 \frac{-\hat{\mu}_{net} \cdot Q_\text{N,0}/2}{f_A \cdot \hat{A}_\text{N,0} \cdot \text{DIN}^2}
\end{flalign} 
But it delivers exactly the same results. I am just wondering how the original solution was found :-)}\\

For calculating the partial derivative of $Z$ with respect to $\bar{I}$, $\frac{\partial Z}{\partial \bar{I}}$, we first make the simplifying assumption that the chlorophyll density in the chloroplasts, $\hat{\theta}$ is constant, such that a complicated (see Eq.~26 in K21) $\frac{\partial \hat{\theta}}{\partial \bar{I}}$ does not have to be taken into account \onur{which seems to have been ignored in S16?}. Recalling $\hat{\mu}_{net}$ and $L_I$ from K21, Eqs.~20-23:
\begin{flalign}
 %\hat{\mu}_{net} &= \hat{\mu}_g (1-\zeta_{Chl}\hat{\theta})- R^{Chl}_\text{M} \zeta_{Chl}\hat{\theta}\\
 \hat{\mu}_{net} &= L_\text{D}\hat{\mu}_0 L_{I} (1-\zeta_{Chl}\hat{\theta})- R^{Chl}_\text{M} \zeta_{Chl}\hat{\theta}\\
  L_I &= 1 - \exp \left( \frac{-\alpha \hat{\theta} \bar{I}}{\hat{\mu}_0} \right)
\end{flalign}
$\frac{\partial Z}{\partial \bar{I}}$ can be then derived as:
\begin{flalign}
 \frac{\partial Z}{\partial \bar{I}} =
 \frac{\partial Z}{\partial \hat{\mu}_{net}} \frac{\partial \hat{\mu}_{net}}{\partial \bar{I}} &=
 \frac{\partial Z}{\partial \hat{\mu}_{net}} \frac{\partial \hat{\mu}_{net}}{\partial L_I} \frac{\partial L_I}{\partial\bar{I}}=
 \frac{Q_\text{N,0}}{2 \cdot \hat{V}_\text{N}} \left(L_\text{D} \cdot (1-\hat{\theta} \cdot \zeta_\text{Chl}) \right) \left ( \alpha \cdot \hat{\theta} \cdot (1-L_\text{I}) \right) 
\end{flalign}
where $\alpha$ and $\zeta_\text{Chl}$ are model parameters (initial Chl-specific slope of P-I curve and cost of chlorophyll synthesis, respectively, Table~3 in K21). %  $\hat{\theta}$ is the optimal chlorophyll density in the chloroplast.\\
We would like to clarify that 1) the DIN in the denominotor in the solution provided by S16 (Eq.~A-5) was a typo; 2) replacement of $\hat{\mu}_g$ (in S16, $\hat{\mu}^I$) with $\hat{\mu}_{net}$ in our model (see K21) results in appearance of $(1-\hat{\theta} \cdot \zeta_\text{Chl})$ when computing $\frac{\hat{\mu}_{net}}{\partial L_I}$; 3) $L_\text{D}$ used to be implicit in S16, now it's explicit (K21 eq. 21) therefore it appears for $\frac{\hat{\mu}_{net}}{\partial L_I}$ unlike in S16 \onur{This is what I'm guessing, but maybe it was just forgotten}. However, since $L_D$ is set to 1 in the current study (Section~\ref{S:DescSetup}), this difference is practically not relevant for our immediate purposes.\\ 
%4)Probably we should have considered $\frac{\partial{\hat{\theta}}}{\partial{\bar{I}}}$ as well, since $\hat{\theta} = f(\bar{I}$ (Eq.~26 in K21), but it appears as if it wasn't in S16 either (Eq.~A-4)}\\
%\onur{Note that in the code, $\frac{\text{DIN}_{i} - \text{DIN}_{i-1}}{\Delta t}$ are still calculated, but are not required in right hand sides, and only saved as diagnostic quantities}

The final term required in Eq.~\ref{eq:sdin3} is the change in $\bar{I}$ over time, i.e., $\frac{\text{d}\bar{I}}{\text{d}t}$.
We consider two cases with regard to the handling of this term:
\begin{description}
 \item [PAR:N] in the first case, short wave radiation is supplied externally, as typically the case in realistic coupled physical-biological models. When the irradiance is supplied externally, it is not possible to obtain its temporal derivative analytically, therefore  it is numerically approximated as the discrete backward difference between its current ($t=i$) and previous ($t=i-1$) values, divided by the size of the integration time step i.e., $\frac{\text{d} \bar{I}}{\text{d} t} \approx \frac{\bar{I}_{i} - \bar{I}_{i-1}}{\Delta t}$. For being able to compare this approach with the analytical approach described below, the externally supplied photosynthetically available irradiance is calculated as a sinusoidal function:
 \begin{flalign}\label{eq.I}
  \bar{I}(t) &= \bar{I}_{\min} + (\bar{I}_{\max}-\bar{I}_{\min}) 0.5 \left[ 1 + \sin \left[ 2 \pi (t/365-0.25) \right]  \right] &&
 \end{flalign}
 where, $\bar{I}_{\min}$ (set to 1.6 [mol m$^2$ d$^{-1}$])  and $\bar{I}_{\max}$ (set to 110 [mol m$^2$ d$^{-1}$]) define the minimum and maximum values throughout the year, and $t$ is the day of the year. $t' = t/365 - 0.25$ represents the relative day of the year delayed by a quarter cycle to obtain the peak value at the middle of the year to represent a seasonal cycle representative of a high latitude environment in the northern hemisphere (see Fig.~\ref{f.I} for the behavior of the function with these parameters). 
 
 \item [PAR:A] with the numerical approach described above, approximating the derivative with a discrete difference term can lead to inaccuracies. In order to diagnose such a potential source of error, we consider a a second case, where the short wave radiation is calculated internally, still according to Eq.\ref{eq.I} as in the approach described above, but for its temporal derivative, an exact, analytical solution is derived. Applying the chain rule with $U=2 \pi t'$:
 \begin{flalign}\label{eq.dIdt}
  \frac{\text{d}\bar{I}}{\text{d}t} &= \frac{\text{d}\bar{I}}{\text{d}U} \frac{\text{d}U}{\text{d}t'}\frac{\text{d}t'}{\text{d}t}&&\\
  \frac{\text{d}\bar{I}}{\text{d}t} &= (\bar{I}_{\max}-\bar{I}_{\min}) \frac{\pi}{365} \cos \left[ 2 \pi (t/365-0.25) \right]  &&
 \end{flalign}
 It can be noted that the temporal derivative of irradiance is positive during the first half of the year, and negative in the second half (Fig.~\ref{f.I}).
\end{description}

%$\bar{I}_{\min}$=1.6 and $\bar{I}_{\max}$=110 

%\begin{figure}[t]
%\includegraphics[width=8.3cm,trim=0cm 6mm 0.0cm 10mm, clip]{figures/fI_analytical.png}
%\caption{Seasonal course of analytically described (Eqs.\ref{f.Ian}) daily average irradiance (a) and its temporal derivative (b), with $\bar{I}_{\min}$=1.6 and $\bar{I}_{\max}$=110 [mol m$^2$ d$^{-1}$]. \label{f.Ian}}
%\end{figure}

%and $\frac{\text{d} \text{DIN}}{\text{d} t} \approx \frac{\text{DIN}_{i} - \text{DIN}_{i-1}}{\Delta t}$.  

%\newpage

% RESULTS
\section{Results}
Both the irradiance, and its temporal derivative, as found by both the numerical and analytical approaches seem to be identical (Fig.~\ref{f.I}).
\begin{figure}[ht!]
\includegraphics[width=14cm,trim=0cm 0mm 0.0cm 0mm, clip]{figures/2022-01-05_T10_TheHat03_Ld1_PARint_vs_PARext/0D-Highlat_wconst_T10_lext_CbasedIA_TheHat03_Ld1_newdelZdelN_mortC_dm_vs_0D-Highlat_wconst_T10_lint_CbasedIA_TheHat03_Ld1_newdelZdelN_mortC_dm_cont_abio0_2y.png}
\caption{Daily average irradiance and its temporal derivative, as extracted from the simulation outputs. PAR:N (solid blue line) is the model version where irradiance is externally supplied, and its derivative is numerically approximated; PAR:A (dashed orange line): both irradiance and its temporal derivative are analytically calculated inside the model . \label{f.I}}
\end{figure}

As shown in Fig.~\ref{f.CNvars}, the model is fully conservative with respect to C, which is as expected, but not with respect to N. The total N follows a seasonal pattern with an increase in the first half of the year, and a decrease in the second half, but not returning back to the starting point at the beginning of the year but at a slightly higher level. The estimated concentrations of individual pools, and the non-conserved total N according to the two approaches are almost identical (note the slightly higher Total N as predicted by the analytical approach). Therefore the possibility that this non-conservative behavior might be related with the numerical approximation of $\text{d}\bar{I}/\text{d}t$  can be ruled out. The matching signs of seasonal changes in total N and $\text{d}\bar{I}/\text{d}t$  (i.e., positive in the first of half, negative in the second), is suggestive of a potential flaw in the term multiplied by $\text{d}\bar{I}/\text{d}t$ in Eq.~\ref{eq:sdin3}, that is,  $\frac{\partial Q_\text{N}}{\partial \bar{I}}$, thus, the individual terms it comprises (Eq.~\ref{eq.delQdelI}), i.e., $\frac{\partial Q_\text{N}}{\partial Z}$ and $\frac{\partial Z}{\partial \bar{I}}$. 
%The most likely flaw is the oversight of the fact that $\hat{\theta}=f(\bar{I})$, which therefore requires consideration of $\frac{\partial \hat{\theta}}{\bar{I}}$.

\begin{figure}[ht!]
\includegraphics[width=14cm,trim=0cm 9mm 0.0cm 5mm, clip]{figures/2022-01-05_T10_TheHat03_Ld1_PARint_vs_PARext/0D-Highlat_wconst_T10_lext_CbasedIA_TheHat03_Ld1_newdelZdelN_mortC_dm_vs_0D-Highlat_wconst_T10_lint_CbasedIA_TheHat03_Ld1_newdelZdelN_mortC_dm_cont_abio1_2y.png}
\caption{Carbon (left) and nitrogen (right) pools according to the PAR:N (solid blue line), and PAR:A (dashed orange line) approaches. \label{f.CNvars}}
\end{figure}



% DISCUSSION
%\section{Discussion}


%CONCLUSION
%\section{Conclusions}


%\codeavailability{For running the model and reproducing the results presented in this study, FABM and GOTM need to be downloaded and installed. See https://github.com/fabm-model/fabm/wiki/GOTM for the instructions. The FABM-NflexPD is available from the ‘Cbased’ branch of the git repository:\url{https://github.com/OnurKerimoglu/fabm-nflexpd.git}. Instructions for compiling FABM-NflexPD for GOTM-FABM and a 0D setup are provided in README.md. The `src’ folder contains the Fortran codes. The model was implemented as two separate modules: the `phy.F90' module that describes phytoplankton growth and the `abio.F90' module that describes everything other than phytoplankton. The phytoplankton module can reproduce the behavior of all three different variants considered in the manuscript through optional parameters. The `testcases’ folder contains the configuration (yaml) file that was used to produce the results presented in this manuscript, thereby providing examples of how each variant can be initiated.} %% use this section when having only software code available


%\dataavailability{TEXT} %% use this section when having only data sets available


%\codedataavailability{TEXT} %% use this section when having data sets and software code available


%\sampleavailability{TEXT} %% use this section when having geoscientific samples available


%\videosupplement{TEXT} %% use this section when having video supplements available

\clearpage

\appendix
\section{Relevance of changes in quota for the loss processes}\label{S:Q4loss}  %% Appendix-A
%In our model, we considered a single loss process for phytoplankton. 
%Recalling from K21 (Table 1) that the associated N flux term was expressed as 
%\begin{flalign}
% F_{\text{Phy}_\text{N}-\text{Det}_\text{N}} &= m \cdot \text{Phy}_\text{N}^2 &&
%\end{flalign}
%and considering that $\text{Phy}_\text{N}=\text{Phy}_\text{C} \cdot Q_\text{N}$, it can be thought that the changes in $Q_\text{N}$ needs to be taken into account for calculating the flux between phytoplankton and detrital N due to mortality. 
Potential links between cellular quotas and loss processes can be assessed by analyzing the components of the changes in $\text{Phy}_\text{N}$ occurring due to loss terms, as was previously done for the case of growth (Eq.~\ref{eq:dphyNdt}). Observing that the only loss process we considered here is mortality:
\begin{flalign}
\frac{\text{d}\text{Phy}_\text{N}}{\text{d}t} \bigg\rvert_L
&=\frac{\text{d}\text{Phy}_\text{C} Q_\text{N}}{\text{d}t} \bigg\rvert_M
= Q_\text{N} \frac{\text{d}\text{Phy}_\text{C}}{\text{d}t} \bigg\rvert_M + \text{Phy}_\text{N}\frac{\text{d}\text{Q}_\text{N}}{\text{d}t} \bigg\rvert_M
\end{flalign}
Here, in the first term, $\frac{\text{d}\text{Phy}_\text{C}}{\text{d}t} \rvert_M$ represents the C-based mortality rate ($=m \cdot \text{Phy}_\text{C}^2$, Eq.~\ref{eq:mortC}). In the second term, $\frac{\text{d}\text{Q}_\text{N}}{\text{d}t} \rvert_M$ represents the changes occurring in $Q_\text{N}$, however, due to mortality only. As in our model, we did not consider any effect of mortality on quota (assuming that when a cell dies, it dies altogether and all of its contents become detritus), this term collapses to 0, therefore reducing the associated flux term to:
\begin{flalign}
F_{\text{Phy}_\text{N}-\text{Det}_\text{N}} &=
Q_\text{N} \frac{\text{d}\text{Phy}_\text{C}}{\text{d}t} \bigg\rvert_M = Q_\text{N} \cdot m \cdot \text{Phy}_\text{C}^2 
\end{flalign}
However, in any other model that may consider processes that may affect the cellular quotas, such as excretion of C-rich, N-poor substances, associated consequences on various elemental fluxes can be taken into account following this framework.

\noappendix%% use this to mark the end of the appendix section. Otherwise the figures might be numbered incorrectly (e.g. 10 instead of 1).

%% Regarding figures and tables in appendices, the following two options are possible depending on your general handling of figures and tables in the manuscript environment:

%% Option 1: If you sorted all figures and tables into the sections of the text, please also sort the appendix figures and appendix tables into the respective appendix sections.
%% They will be correctly named automatically.

%% Option 2: If you put all figures after the reference list, please insert appendix tables and figures after the normal tables and figures.
%% To rename them correctly to A1, A2, etc., please add the following commands in front of them:

%\appendixfigures%% needs to be added in front of appendix figures

%\appendixtables%% needs to be added in front of appendix tables

%% Please add \clearpage between each table and/or figure. Further guidelines on figures and tables can be found below.

%\authorcontribution{} %% this section is mandatory

%\competinginterests{No competing interests are present.} %% this section is mandatory even if you declare that no competing interests are present

%\disclaimer{TEXT} %% optional section

%\begin{acknowledgements}
%OK, SLS and PA were supported through a bilateral research project, funded jointly by the German Research Foundation DFG (grant no. KE 1970/2-1, P.I.: OK) and the Japan Society for the Promotion of Science, JSPS (P.I.: SLS).We acknowledge the developers of the open-source software used in this study, foremost FABM and GOTM.
%\end{acknowledgements}



%% REFERENCES

%% The reference list is compiled as follows:

% \begin{thebibliography}{}
% 
% \bibitem[AUTHOR(YEAR)]{LABEL1}
% REFERENCE 1
% 
% \bibitem[AUTHOR(YEAR)]{LABEL2}
% REFERENCE 2
% 
% \end{thebibliography}

%% Since the Copernicus LaTeX package includes the BibTeX style file copernicus.bst,
%% authors experienced with BibTeX only have to include the following two lines:
%%
\bibliographystyle{copernicus}
\bibliography{NflexPD_2_0.bib}
%%
%% URLs and DOIs can be entered in your BibTeX file as:
%%
%% URL = {http://www.xyz.org/~jones/idx_g.htm}
%% DOI = {10.5194/xyz}


%% LITERATURE CITATIONS
%%
%% command                        & example result
%% \citet{jones90}|               & Jones et al. (1990)
%% \citep{jones90}|               & (Jones et al., 1990)
%% \citep{jones90,jones93}|       & (Jones et al., 1990, 1993)
%% \citep[p.~32]{jones90}|        & (Jones et al., 1990, p.~32)
%% \citep[e.g.,][]{jones90}|      & (e.g., Jones et al., 1990)
%% \citep[e.g.,][p.~32]{jones90}| & (e.g., Jones et al., 1990, p.~32)
%% \citeauthor{jones90}|          & Jones et al.
%% \citeyear{jones90}|            & 1990



%% FIGURES

%% When figures and tables are placed at the end of the MS (article in one-column style), please add \clearpage
%% between bibliography and first table and/or figure as well as between each table and/or figure.

% The figure files should be labelled correctly with Arabic numerals (e.g. fig01.jpg, fig02.png).


%% ONE-COLUMN FIGURES

%%f
%\begin{figure}[t]
%\includegraphics[width=8.3cm]{FILE NAME}
%\caption{TEXT}
%\end{figure}
%
%%% TWO-COLUMN FIGURES
%
%%f
%\begin{figure*}[t]
%\includegraphics[width=12cm]{FILE NAME}
%\caption{TEXT}
%\end{figure*}
%
%
%%% TABLES
%%%
%%% The different columns must be seperated with a & command and should
%%% end with \\ to identify the column brake.
%
%%% ONE-COLUMN TABLE
%
%%t
%\begin{table}[t]
%\caption{TEXT}
%\begin{tabular}{column = lcr}
%\tophline
%
%\middlehline
%
%\bottomhline
%\end{tabular}
%\belowtable{} % Table Footnotes
%\end{table}
%
%%% TWO-COLUMN TABLE
%
%%t
%\begin{table*}[t]
%\caption{TEXT}
%\begin{tabular}{column = lcr}
%\tophline
%
%\middlehline
%
%\bottomhline
%\end{tabular}
%\belowtable{} % Table Footnotes
%\end{table*}
%
%%% LANDSCAPE TABLE
%
%%t
%\begin{sidewaystable*}[t]
%\caption{TEXT}
%\begin{tabular}{column = lcr}
%\tophline
%
%\middlehline
%
%\bottomhline
%\end{tabular}
%\belowtable{} % Table Footnotes
%\end{sidewaystable*}
%
%
%%% MATHEMATICAL EXPRESSIONS
%
%%% All papers typeset by Copernicus Publications follow the math typesetting regulations
%%% given by the IUPAC Green Book (IUPAC: Quantities, Units and Symbols in Physical Chemistry,
%%% 2nd Edn., Blackwell Science, available at: http://old.iupac.org/publications/books/gbook/green_book_2ed.pdf, 1993).
%%%
%%% Physical quantities/variables are typeset in italic font (t for time, T for Temperature)
%%% Indices which are not defined are typeset in italic font (x, y, z, a, b, c)
%%% Items/objects which are defined are typeset in roman font (Car A, Car B)
%%% Descriptions/specifications which are defined by itself are typeset in roman font (abs, rel, ref, tot, net, ice)
%%% Abbreviations from 2 letters are typeset in roman font (RH, LAI)
%%% Vectors are identified in bold italic font using \vec{x}
%%% Matrices are identified in bold roman font
%%% Multiplication signs are typeset using the LaTeX commands \times (for vector products, grids, and exponential notations) or \cdot
%%% The character * should not be applied as mutliplication sign
%
%
%%% EQUATIONS
%
%%% Single-row equation
%
%\begin{equation}
%
%\end{equation}
%
%%% Multiline equation
%
%\begin{align}
%& 3 + 5 = 8\\
%& 3 + 5 = 8\\
%& 3 + 5 = 8
%\end{align}
%
%
%%% MATRICES
%
%\begin{matrix}
%x & y & z\\
%x & y & z\\
%x & y & z\\
%\end{matrix}
%
%
%%% ALGORITHM
%
%\begin{algorithm}
%\caption{...}
%\label{a1}
%\begin{algorithmic}
%...
%\end{algorithmic}
%\end{algorithm}
%
%
%%% CHEMICAL FORMULAS AND REACTIONS
%
%%% For formulas embedded in the text, please use \chem{}
%
%%% The reaction environment creates labels including the letter R, i.e. (R1), (R2), etc.
%
%\begin{reaction}
%%% \rightarrow should be used for normal (one-way) chemical reactions
%%% \rightleftharpoons should be used for equilibria
%%% \leftrightarrow should be used for resonance structures
%\end{reaction}
%
%
%%% PHYSICAL UNITS
%%%
%%% Please use \unit{} and apply the exponential notation

\end{document}

% Local Variables:
% TeX-engine: default
% End:
