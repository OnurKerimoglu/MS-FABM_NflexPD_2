%% Copernicus Publications Manuscript Preparation Template for LaTeX Submissions
%% ---------------------------------
%% This template should be used for copernicus.cls
%% The class file and some style files are bundled in the Copernicus Latex Package, which can be downloaded from the different journal webpages.
%% For further assistance please contact Copernicus Publications at: production@copernicus.org
%% https://publications.copernicus.org/for_authors/manuscript_preparation.html


%% Please use the following documentclass and journal abbreviations for preprints and final revised papers.

%% 2-column papers and preprints
\documentclass[gmd, manuscript]{copernicus}
%\documentclass[gmd, manuscript, draft]{copernicus}

%% Journal abbreviations (please use the same for preprints and final revised papers)

% Advances in Geosciences (adgeo)
% Advances in Radio Science (ars)
% Advances in Science and Research (asr)
% Advances in Statistical Climatology, Meteorology and Oceanography (ascmo)
% Annales Geophysicae (angeo)
% Archives Animal Breeding (aab)
% ASTRA Proceedings (ap)
% Atmospheric Chemistry and Physics (acp)
% Atmospheric Measurement Techniques (amt)
% Biogeosciences (bg)
% Climate of the Past (cp)
% DEUQUA Special Publications (deuquasp)
% Drinking Water Engineering and Science (dwes)
% Earth Surface Dynamics (esurf)
% Earth System Dynamics (esd)
% Earth System Science Data (essd)
% E&G Quaternary Science Journal (egqsj)
% European Journal of Mineralogy (ejm)
% Fossil Record (fr)
% Geochronology (gchron)
% Geographica Helvetica (gh)
% Geoscience Communication (gc)
% Geoscientific Instrumentation, Methods and Data Systems (gi)
% Geoscientific Model Development (gmd)
% History of Geo- and Space Sciences (hgss)
% Hydrology and Earth System Sciences (hess)
% Journal of Bone and Joint Infection (jbji)
% Journal of Micropalaeontology (jm)
% Journal of Sensors and Sensor Systems (jsss)
% Magnetic Resonance (mr)
% Mechanical Sciences (ms)
% Natural Hazards and Earth System Sciences (nhess)
% Nonlinear Processes in Geophysics (npg)
% Ocean Science (os)
% Primate Biology (pb)
% Proceedings of the International Association of Hydrological Sciences (piahs)
% Scientific Drilling (sd)
% SOIL (soil)
% Solid Earth (se)
% The Cryosphere (tc)
% Weather and Climate Dynamics (wcd)
% Web Ecology (we)
% Wind Energy Science (wes)


%% \usepackage commands included in the copernicus.cls:
%\usepackage[german, english]{babel}
%\usepackage{tabularx}
%\usepackage{cancel}
%\usepackage{multirow}
%\usepackage{supertabular}
%\usepackage{algorithmic}
%\usepackage{algorithm}
%\usepackage{amsthm}
%\usepackage{float}
%\usepackage{subfig}
%\usepackage{rotating}

%Additional packages: 
\usepackage{mathrsfs} %provides calligraphic Fonts
\allowdisplaybreaks

\graphicspath{
%{./figures/common/}
%{./figures/v_10_13_solitary/}
{./figures/}
}

\newcommand{\onur}[1]{\textcolor{blue}{\{Onur: #1\}}}

\begin{document}
\title{FABM-NflexPD 2.0: An instantenous acclimation approach for modelling carbon, nitrogen and phosphorus cycles}

% \Author[affil]{given_name}{surname}

\Author[1,2]{Onur}{Kerimoglu}
\Author[3]{Prima}{Anugerahanti}
\Author[3]{S. Lan}{Smith}

\affil[1]{Institute for Chemistry and Biology of the Marine Environment, University of Oldenburg, Germany}
\affil[2]{Helmholtz Center for Coastal Research, Germany}
\affil[3]{Earth SURFACE Research Center, Research Institute for Global Change, JAMSTEC, Japan}

%% The [] brackets identify the author with the corresponding affiliation. 1, 2, 3, etc. should be inserted.

%% If an author is deceased, please mark the respective author name(s) with a dagger, e.g. "\Author[2,$\dag$]{Anton}{Aman}", and add a further "\affil[$\dag$]{deceased, 1 July 2019}".

%% If authors contributed equally, please mark the respective author names with an asterisk, e.g. "\Author[2,*]{Anton}{Aman}" and "\Author[3,*]{Bradley}{Bman}" and add a further affiliation: "\affil[*]{These authors contributed equally to this work.}".


\correspondence{Onur Kerimoglu (kerimoglu.o@gmail.com)}

\runningtitle{FABM-NflexPD 2.0}

\runningauthor{Kerimoglu et al.}


\received{}
\pubdiscuss{} %% only important for two-stage journals
\revised{}
\accepted{}
\published{}

%% These dates will be inserted by Copernicus Publications during the typesetting process.


\firstpage{1}

\maketitle


\begin{abstract}

\end{abstract}


%\copyrightstatement{TEXT}

%INTRODUCTION
\introduction%% \introduction[modified heading if necessary]

%\subsection{Autotroph Acclimation and Biogeochemical Cycles} 

%\subsection{Instantaneous acclimation approach}
\subsection{Background and Objectives}
Recently, \citet{Kerimoglu2021} introduced FABM-NflexPD 1.0, which is the FABM \citep[Framework for Aquatic Biogeochemical Models][]{Bruggeman2014} implementation of an instantaneous acclimation model of phytoplankton growth originally developed by \citet{Smith2016}. That model version can only resolve N in the Dissolved Inorganic Material (DIM) pool, which may be sufficient for some ecological applications, but not for applications that require resolution of multiple nutirents in DIM form.

%\subsection{Objectives of this study}
Here we introduce FABM-NflexPD 2.0, which can resolve C, N and P in DIM pool. We present:
\begin{itemize}
 \item a detailed description of the current, C-based version with reference to the previous, N-based version (see below) 
 \item comparison of the behavior of the previous and current C- and N- based model versions in a 0D setup
 \item description of the model extended with P limitation
 \item application of the C-N-P model to an ocean site in a 1D setup
\end{itemize}

% \begin{figure}[htb]
%   \centering
%   \includegraphics[width=12cm]{?.pdf}
%   \caption{\label{?}}
% \end{figure}

%MODEL DESCRIPTION
\section{Model Description}
In this section, we will highlight the differences and similarities between the present, `C-based' version of the model from the earlier version of the `N-based' model version presented by \citet{Kerimoglu2021}, K21-N in short hereafter.

\subsection{C and N content of Phytoplankton}
In this version of the model, C content of phytoplankton ($Phy_\text{C}$) is dynamically tracked (thus, C-based) instead of the N content ($Phy_\text{N}$) as was the case in the earlier (thus, N-based) version. Wherever required, $Phy_\text{N}$ is given by the product of $Phy_\text{C}$ and N quota, $Q_\text{N}$:
\begin{equation} \label{eq:Q}
Phy_\text{N} = Phy_\text{C} \cdot Q_\text{N}  
\end{equation}

$Q_\text{N}$ is assumed to instantaneously adjust to a balanced-growth optimum value, as determined by the nutrient uptake in the protopolast, $\hat{V}_\text{N}$ and net photosynthesis in the chloroplast, $\hat{\mu}_{net}$ (Eq.~10 in K21-N):
\begin{equation}\label{eq:Qopt}
 Q_\text{N}^o= \frac{Q_\text{N,0}}{2} \left[1+\sqrt{1+\frac{2}{Q_\text{N,0}{({\hat{\mu}_{\text{net}}}/{\hat{V}_\text{N}}+\zeta_{\text{N}} )}}} \right] \hfill \{\text{IA}\}
\end{equation}
where $Q_\text{N,0}$ is the subsistence N quota (K21-N, Table~3). The source term for $Phy_\text{C}$ is given by:
\begin{equation} \label{eq:sPhyC}
s(Phy_{\text{C}}) = F_{DIC-Phy_\text{C}} - F_{Phy_{\text{C}}-Det_{\text{C}}}
\end{equation}
which is identical to the source term of the `DA' variant (Eq.~1b in K21). Here, $F_{DIC-Phy_\text{C}}$ and $F_{Phy_{\text{C}}-Det_{\text{C}}}$ are the fluxes from Dissolved Inorganic Carbon ($DIC$) to phytoplankton, and from phytoplankton to detrital C ($Det_\text{C}$), respectively. The first term in Eq.~\ref{eq:sPhyC} represents net phytoplankton growth, and as before (Eq.~8 in K21-N), it is given by the product of $Phy_\text{C}$ with net growth rate, $\mu$: 
\begin{equation} \label{eq:fdicphyc}
 F_{DIC-Phy_\text{C}} = \mu \cdot Phy_\text{C}
\end{equation}
Calculation of $\mu$, in turn, is as described earlier (Eq.~7 in K21-N). The second term in Eq.~\ref{eq:sPhyC}, $F_{Phy_\text{C}-Det_\text{C}}$, representing the mortality of phytoplankton, is the same as in the N-based version, i.e., it is given by the division of N flux from phytoplankton to detritus, $F_{Phy_\text{N}-Det_\text{N}}$ by $Q_N$ (Table 1 in K21-N).

\subsection{Flux of nutrients between DIM and Phytoplankton}
As in K21-N, the source term for the nutrients in DIM pool are described as the fluxes from the nutrients Dissolved Organic form (e.g., DON), $F_{DON-DIN}$ (Table 1, K21-N), minus the fluxes between the DIM (e.g., DIN) and phytoplankton ($F_{DIN-Phy_\text{N}}$): 
\begin{equation} \label{eq:sdin}
  s(DIN) = \frac{\text{d}\text{DIN}}{\text{d}t} = F_{DON-DIN} - F_{DIN-Phy_\text{N}}
\end{equation}

$F_{DON-DIN}$ is calculated exactly as described in K21-N. $F_{DIN-Phy_\text{N}}$, in other words, change in N-biomass of phytoplankton excluding mortality, i.e., $\frac{\text{d}Phy_\text{N}}{\text{d}t} \big\rvert_G$, can be expressed as:
\begin{equation} \label{eq:dphyNdt}
  F_{DIN-Phy_\text{N}} = \frac{\text{d}Phy_\text{N}}{\text{d}t} \bigg\rvert_G=\frac{\text{d}(Phy_\text{C} \cdot Q)}{\text{d}t} \bigg\rvert_G = Q_\text{N} \cdot \frac{\partial Phy_\text{C}}{\partial t} \bigg\rvert_G + Phy_\text{C} \cdot \frac{\partial Q_\text{N}}{\partial t} 
\end{equation}

Here, $\frac{\partial Phy_\text{C}}{\partial t} \big\rvert_G$ corresponds to change in C-biomass of phytoplankton excluding mortality, i.e., $F_{DIC-Phy_\text{C}}$ in Eq.~\ref{eq:sPhyC}. Substituting therefore Eq.~\ref{eq:fdicphyc} in Eq.~\ref{eq:dphyNdt}:
\begin{equation} \label{eq:dphyNdt2}
  \frac{\text{d}Phy_\text{N}}{\text{d}t} \bigg\rvert_G= Q_\text{N} \cdot \mu \cdot Phy_\text{C} + Phy_\text{C} \cdot \frac{\partial Q_\text{N}}{\partial t} 
\end{equation}

where, assuming balanced growth \citep{Burmaster1979}, i.e., $V_N = Q_N \cdot \mu$  (K21-N, Eq.~6) the first term on the right hand side of Eq.~\ref{eq:dphyNdt2} can be replaced with $V_N \cdot Phy_\text{C}$, i.e., 
\begin{equation} \label{eq:dphyNdt3}
  \frac{\text{d}Phy_\text{N}}{\text{d}t}=  Phy_\text{C} \left( V_\text{N} + \frac{\partial Q_\text{N}}{\partial t} \right)
\end{equation}

\onur{In the N-based version, $F_{DIN-Phy_\text{N}}$ was given by (K21-N, Eq.~5):
\begin{equation}\label{eq:Fdinphy}
  F_{DIN-Phy_{\text{N}}} \bigg\rvert_G = V_N \cdot Phy_{\text{C}}
\end{equation}
i.e., excluding the $\frac{\partial Q_\text{N}}{\partial t}$ term in Eq.~\ref{eq:dphyNdt3}. Does this make sense? Or what am I missing? We must have discussed this back when I was in Japan, but I cannot recall anymore. 
}

Given that the terms $\hat{V}_\text{N}$ and $\hat{mu}_net$ used for calculating $Q_\text{N}$ (Eq.~\ref{eq:Qopt}) are respectively functions of DIN (Eq.~16 in K21-N), $\bar{I}$ and daytime average irradiance, $\bar{I}$ (Eqs.~20-22 in K21-N), $\frac{\partial Q_\text{N}}{\partial t}$ in Eq.~\ref{eq:dphyNdt2} :
\begin{equation} \label{eq:dqdt}
 \frac{\partial Q_\text{N}}{\partial t} = \frac{\partial Q_\text{N}}{\partial \text{DIN}} \frac{\text{d} \text{DIN}}{\text{d} t} +  \frac{\partial Q_\text{N}}{\partial \bar{I}} \frac{\text{d} \bar{I}}{\text{d} t} 
\end{equation}

In Eq.~\ref{eq:Qopt}, defining $Z = \frac{2}{Q_\text{N,0}{({\hat{\mu}_{\text{net}}}/{\hat{V}_\text{N}}+\zeta_{\text{N}})}}$ following \citet{Smith2016}, partial derivatives of $Q_\text{N}$ with respect to DIN and $\bar{I}$ can be found as:
\begin{flalign}
 \frac{\partial Q_\text{N}}{\partial \text{DIN}} &= \frac{-Q_\text{N,0}}{4 \cdot Z \cdot \sqrt{Z\cdot(1+Z)}} \cdot \frac{\partial Z}{\partial \text{DIN}}\\
 \frac{\partial Q_\text{N}}{\partial \bar{I}}  &= \frac{-Q_\text{N,0}}{4 \cdot Z \cdot \sqrt{Z\cdot(1+Z)}} \cdot \frac{\partial Z}{\partial \bar{I}}
\end{flalign}
and the derivatives of $Z$:
\begin{flalign}
 \frac{\partial Z}{\partial \text{DIN}} &= \frac{-\hat{\mu}_{net} \cdot Q_\text{N,0}}{2 \cdot \hat{V}_\text{N} \cdot \text{DIN}} \cdot \left( 1 - \frac{\hat{V}_\text{N}}{\hat{V}_\text{N,0}} - \frac{\hat{V}_\text{N}}{\sqrt{\hat{V}_\text{N,0} \cdot \hat{A}_\text{N,0} \cdot \text{DIN}}} \right) \\
 \frac{\partial Z}{\partial \bar{I}} &= \frac{Q_\text{N,0} \cdot \alpha \cdot \hat{\Theta}}{\hat{V}_\text{N} \cdot \text{DIN}} \cdot \left ( 1 - L_\text{I} \right) 
\end{flalign}
where \onur{check: replacement of $\hat{\mu}_g$ (in S16, $\hat{\mu}^I$) with $\hat{\mu}_{net}$ may require revision of last equation}.

The last two terms required in Eq.~\ref{eq:dqdt} are the changes in $\bar{I}$ and DIN over time. The former, $\frac{\text{d} \bar{I}}{\text{d} t}$ is approximated by the discrete backward difference between the current ($t=i$) and previous ($t=i-1$) value of $\bar{I}$, divided by the size of the integration time step i.e., $\frac{\text{d} \bar{I}}{\text{d} t} \sim \frac{\bar{I}_{i} - \bar{I}_{i-1}}{\Delta t}$.  The latter, $\frac{\text{d} DIN}{\text{d} t}$ is eliminated by first doing the necessary substitutions in Eq.~\ref{eq:sdin}:
\begin{equation}\label{eq:sdin2}
 \frac{\text{d}\text{DIN}}{\text{d}t} = F_{DON-DIN} - Phy_{\text{C}} \left(V_N + \frac{\partial Q_\text{N}}{\partial \text{DIN}} \frac{\text{d} \text{DIN}}{\text{d} t} +  \frac{\partial Q_\text{N}}{\partial \bar{I}} \frac{\text{d} \bar{I}}{\text{d} t} \right)
\end{equation}
and reorganizing:
%intermediate step:
% \begin{equation}\label{eq:sdin2.5}
%  \frac{\text{d}\text{DIN}}{\text{d}t} \left( 1+Phy_{\text{C}}\frac{\partial Q_\text{N}}{\partial \text{DIN}} \right) = F_{DON-DIN} - Phy_{\text{C}} \left(V_N +  \frac{\partial Q_\text{N}}{\partial \bar{I}} \frac{\text{d} \bar{I}}{\text{d} t} \right)\\
% \end{equation}
\begin{equation}\label{eq:sdin3}
 \frac{\text{d}\text{DIN}}{\text{d}t} = \frac{F_{DON-DIN} - Phy_{\text{C}} \left(V_N +  \frac{\partial Q_\text{N}}{\partial \bar{I}} \frac{\text{d} \bar{I}}{\text{d} t} \right)}{ 1+Phy_{\text{C}}\frac{\partial Q_\text{N}}{\partial \text{DIN}}}
\end{equation}

Eq.~\ref{eq:sdin3} requires combination of terms, which, in K21-N, used to be calculated by separate abiotic ($F_{DON-DIN}$) and phytoplankton modules (all other terms), therefore in the current implementation, all calculations are done in a single module.

% RESULTS
%\section{Results}


% DISCUSSION
%\section{Discussion}


%CONCLUSION
%\section{Conclusions}


%\codeavailability{For running the model and reproducing the results presented in this study, FABM and GOTM need to be downloaded and installed. See https://github.com/fabm-model/fabm/wiki/GOTM for the instructions. The FABM-NflexPD is available from the ‘Cbased’ branch of the git repository:\url{https://github.com/OnurKerimoglu/fabm-nflexpd.git}. Instructions for compiling FABM-NflexPD for GOTM-FABM and a 0D setup are provided in README.md. The `src’ folder contains the Fortran codes. The model was implemented as two separate modules: the `phy.F90' module that describes phytoplankton growth and the `abio.F90' module that describes everything other than phytoplankton. The phytoplankton module can reproduce the behavior of all three different variants considered in the manuscript through optional parameters. The `testcases’ folder contains the configuration (yaml) file that was used to produce the results presented in this manuscript, thereby providing examples of how each variant can be initiated.} %% use this section when having only software code available


%\dataavailability{TEXT} %% use this section when having only data sets available


%\codedataavailability{TEXT} %% use this section when having data sets and software code available


%\sampleavailability{TEXT} %% use this section when having geoscientific samples available


%\videosupplement{TEXT} %% use this section when having video supplements available

%\appendix
%\section{Details of Derivations}  %% Appendix-A


\noappendix%% use this to mark the end of the appendix section. Otherwise the figures might be numbered incorrectly (e.g. 10 instead of 1).

%% Regarding figures and tables in appendices, the following two options are possible depending on your general handling of figures and tables in the manuscript environment:

%% Option 1: If you sorted all figures and tables into the sections of the text, please also sort the appendix figures and appendix tables into the respective appendix sections.
%% They will be correctly named automatically.

%% Option 2: If you put all figures after the reference list, please insert appendix tables and figures after the normal tables and figures.
%% To rename them correctly to A1, A2, etc., please add the following commands in front of them:

%\appendixfigures%% needs to be added in front of appendix figures

%\appendixtables%% needs to be added in front of appendix tables

%% Please add \clearpage between each table and/or figure. Further guidelines on figures and tables can be found below.

%\authorcontribution{} %% this section is mandatory

%\competinginterests{No competing interests are present.} %% this section is mandatory even if you declare that no competing interests are present

%\disclaimer{TEXT} %% optional section

%\begin{acknowledgements}
%OK, SLS and PA were supported through a bilateral research project, funded jointly by the German Research Foundation DFG (grant no. KE 1970/2-1, P.I.: OK) and the Japan Society for the Promotion of Science, JSPS (P.I.: SLS).We acknowledge the developers of the open-source software used in this study, foremost FABM and GOTM.
%\end{acknowledgements}



%% REFERENCES

%% The reference list is compiled as follows:

% \begin{thebibliography}{}
% 
% \bibitem[AUTHOR(YEAR)]{LABEL1}
% REFERENCE 1
% 
% \bibitem[AUTHOR(YEAR)]{LABEL2}
% REFERENCE 2
% 
% \end{thebibliography}

%% Since the Copernicus LaTeX package includes the BibTeX style file copernicus.bst,
%% authors experienced with BibTeX only have to include the following two lines:
%%
\bibliographystyle{copernicus}
\bibliography{NflexPD_1_0.bib}
%%
%% URLs and DOIs can be entered in your BibTeX file as:
%%
%% URL = {http://www.xyz.org/~jones/idx_g.htm}
%% DOI = {10.5194/xyz}


%% LITERATURE CITATIONS
%%
%% command                        & example result
%% \citet{jones90}|               & Jones et al. (1990)
%% \citep{jones90}|               & (Jones et al., 1990)
%% \citep{jones90,jones93}|       & (Jones et al., 1990, 1993)
%% \citep[p.~32]{jones90}|        & (Jones et al., 1990, p.~32)
%% \citep[e.g.,][]{jones90}|      & (e.g., Jones et al., 1990)
%% \citep[e.g.,][p.~32]{jones90}| & (e.g., Jones et al., 1990, p.~32)
%% \citeauthor{jones90}|          & Jones et al.
%% \citeyear{jones90}|            & 1990



%% FIGURES

%% When figures and tables are placed at the end of the MS (article in one-column style), please add \clearpage
%% between bibliography and first table and/or figure as well as between each table and/or figure.

% The figure files should be labelled correctly with Arabic numerals (e.g. fig01.jpg, fig02.png).


%% ONE-COLUMN FIGURES

%%f
%\begin{figure}[t]
%\includegraphics[width=8.3cm]{FILE NAME}
%\caption{TEXT}
%\end{figure}
%
%%% TWO-COLUMN FIGURES
%
%%f
%\begin{figure*}[t]
%\includegraphics[width=12cm]{FILE NAME}
%\caption{TEXT}
%\end{figure*}
%
%
%%% TABLES
%%%
%%% The different columns must be seperated with a & command and should
%%% end with \\ to identify the column brake.
%
%%% ONE-COLUMN TABLE
%
%%t
%\begin{table}[t]
%\caption{TEXT}
%\begin{tabular}{column = lcr}
%\tophline
%
%\middlehline
%
%\bottomhline
%\end{tabular}
%\belowtable{} % Table Footnotes
%\end{table}
%
%%% TWO-COLUMN TABLE
%
%%t
%\begin{table*}[t]
%\caption{TEXT}
%\begin{tabular}{column = lcr}
%\tophline
%
%\middlehline
%
%\bottomhline
%\end{tabular}
%\belowtable{} % Table Footnotes
%\end{table*}
%
%%% LANDSCAPE TABLE
%
%%t
%\begin{sidewaystable*}[t]
%\caption{TEXT}
%\begin{tabular}{column = lcr}
%\tophline
%
%\middlehline
%
%\bottomhline
%\end{tabular}
%\belowtable{} % Table Footnotes
%\end{sidewaystable*}
%
%
%%% MATHEMATICAL EXPRESSIONS
%
%%% All papers typeset by Copernicus Publications follow the math typesetting regulations
%%% given by the IUPAC Green Book (IUPAC: Quantities, Units and Symbols in Physical Chemistry,
%%% 2nd Edn., Blackwell Science, available at: http://old.iupac.org/publications/books/gbook/green_book_2ed.pdf, 1993).
%%%
%%% Physical quantities/variables are typeset in italic font (t for time, T for Temperature)
%%% Indices which are not defined are typeset in italic font (x, y, z, a, b, c)
%%% Items/objects which are defined are typeset in roman font (Car A, Car B)
%%% Descriptions/specifications which are defined by itself are typeset in roman font (abs, rel, ref, tot, net, ice)
%%% Abbreviations from 2 letters are typeset in roman font (RH, LAI)
%%% Vectors are identified in bold italic font using \vec{x}
%%% Matrices are identified in bold roman font
%%% Multiplication signs are typeset using the LaTeX commands \times (for vector products, grids, and exponential notations) or \cdot
%%% The character * should not be applied as mutliplication sign
%
%
%%% EQUATIONS
%
%%% Single-row equation
%
%\begin{equation}
%
%\end{equation}
%
%%% Multiline equation
%
%\begin{align}
%& 3 + 5 = 8\\
%& 3 + 5 = 8\\
%& 3 + 5 = 8
%\end{align}
%
%
%%% MATRICES
%
%\begin{matrix}
%x & y & z\\
%x & y & z\\
%x & y & z\\
%\end{matrix}
%
%
%%% ALGORITHM
%
%\begin{algorithm}
%\caption{...}
%\label{a1}
%\begin{algorithmic}
%...
%\end{algorithmic}
%\end{algorithm}
%
%
%%% CHEMICAL FORMULAS AND REACTIONS
%
%%% For formulas embedded in the text, please use \chem{}
%
%%% The reaction environment creates labels including the letter R, i.e. (R1), (R2), etc.
%
%\begin{reaction}
%%% \rightarrow should be used for normal (one-way) chemical reactions
%%% \rightleftharpoons should be used for equilibria
%%% \leftrightarrow should be used for resonance structures
%\end{reaction}
%
%
%%% PHYSICAL UNITS
%%%
%%% Please use \unit{} and apply the exponential notation

\end{document}

% Local Variables:
% TeX-engine: default
% End:
